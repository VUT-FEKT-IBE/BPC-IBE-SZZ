\section[Software jako předmět právní ochrany\,--\,srovnání ochrany autorskoprávní a patentové ochrany]{Software jako předmět právní ochrany\,--\,srovnání och-rany autorskoprávní a patentové ochrany}

\textbf{Software} je souhrnný název pro všechny počítačové programy používané v počítači, které provádějí nějakou činost.

\textbf{Počítačový program} neví v českém právu přesně definován, ale popisuje se jako \uv{program v jakékoliv formě, včetně těch, které jsou součástí technického vybavení (HW)}. Jak Evropský patentový úřad tak SK jej vyjadřuje jako \uv{serii instrukcí, kterou lze spustit na počítači}.

Autor má práva vycházející z autorských práv, kde jsou tyto práva chráněna jako \textbf{literární dílo} bez ohledu na formu vyjádření.

Ochrana pokrývá vyjádření ve formě:
\begin{itemize}[noitemsep]
    \item strojový kód
    \item zdrojový kód
    \item jejich mezistupně
    \item přípravné koncepční materiály vzniklé při vývoji
    \begin{itemize}[noitemsep]
        \item model architektury softwaru
        \item funkční specifikace
        \item apod.
    \end{itemize}
\end{itemize}

Autorským právem nejsou chráněny myšlenky a principy na nichž je založen jakýkoliv prvek počítačového programu, včetně těch, které jsou podkladem jeho propojení z jiným programem. Takže není chráněna funkcionalita programu, ale pouze její objektivní vyjádření v podobě kódu.

Autorské právo rozlišuje \textbf{osobnostní a majetková práva}. Do osobnostních spadá možnost rozhodnou o zveřejnění díla, osobovat si autorství, rozhodnout o nedotknutelnosti díla. Autor se nemůže vzdát osobnostních práv, jelikož jsou nepřeveditelná. Do majetkových práv spadá rozhodnutí o užití díla (rozmnožení, rozšiřovaní, pronájem, půjčovaní aj.). Osobnostní jsou popsána v \href{https://www.zakonyprolidi.cz/cs/2000-121#p11}{Autorský zákon §11} a majetková  \href{https://www.zakonyprolidi.cz/cs/2000-121#p12}{Autorský zákon §12}, kdy jsou v dalších paragrafech popsány jednotlivé typy užití. 

Jak už bylo zmíněno \textbf{autorskoprávní ochrana} programů je neschopna chránit funkcionalitu daného programu. Dokáže chránit pouze objektivní vyjádření v~kódu, popřípadě jeho vizuální stránku, ale samostatná funkcionalita není chráněna. K přiměřené ochraně samostatné funkcionality je použita \textbf{patentová ochrana}. Patentově nelze chránit počítačové programy ale tzv. vynálezy uskutečňované počítačem.
\newline

\noindent Podmínky patentovatelnosti:
\begin{itemize}[noitemsep]
    \item vynález (ne program!) realizovaný počítačem
    \item novost -- vynález se považuje za~nový, není-li součástí stavu techniky
    \item výsledek vynálezecké činnosti
    \item průmyslová využitelnost -- může-li být vynález vyráběn nebo využíván ve všech odvětvích průmyslu
\end{itemize}
Obecně v~ČR je software chráněn pouze autorským zákonem, pokud nespadá pod \textbf{vynálezy realizované počítačem}, ty lze chránit pomocí patentů.  
Vynálezy uskutečňované počítačem jsou chráněny i v ČR. Dle zákona nelze patentovat \uv{\emph{plány, pravidla a způsob vykonávaní duševní činnosti, hraní her nebo vykonávání obchodní činnosti, jakož i programy počítačů}}.

Do vynálezu realizovaného počítačem lze zahrnout počítačový program myšlený jako produkt. Podmínkou je technický charakter příslušného vynálezu. Počítačový program je vynález realizovaný počítačem, jestliže je schopný vyvolat dodatečný technický účinek, když běží na počítači nebo je na něm nahrán. Nesmí se jednat o běžnou interakci mezi SW a HW. Nelze patentovat software, jako takový ale musí splňovat určité podmínky, jejich společným jmenovatelem je přítomnost dalšího technického prvku. 

Hlavní rozdíly mezi autorským právem a patentem.

\begin{center}
    \begin{tabular}{|l|l|}
    \hline
        Autorské právo & Patent \\\hline
        Získaní je automatické & Musí se o něj požádat\\\hline
        Zaniká 70 let po smrti autora & Vydává se na určitou dobu\\\hline
        Ochrana jako dílo literární & Ochraňuje myšlenky, metody, postupy\\\hline
        Neváže se s žádným poplatkem & Na každé podání se váže poplatek.\\\hline
        Zamezuje neoprávněnému šíření & \\\hline
    \end{tabular}
\end{center}







\newpage
\section{Autorství software\,--\,autor, spoluautor, kolektivní dílo, souborné dílo, odvozené dílo, otázka autorství umělé inteligence}

\newpage
\section{Osobnostní a majetková práva autora software -- obsah, doba trvání, vyčerpání práv}

\newpage
\section[Omezení rozsahu práv autora k software\,--\,dekompilace, reverzní inženýrství a další]{Omezení rozsahu práv autora k software\,--\,dekompi-lace, reverzní inženýrství a další}

\newpage
\section{Ochrana neliterárních složek software\,--\,ochrana funkcionality; ochrana datových, grafických uživatelských a aplikačních programových rozhraní)}

\newpage
\section{Software jako zaměstnanecké dílo, školní dílo, kolektivní dílo a dílo na objednávku}

\newpage
\section{Právní ochrana databází\,--\,pojem, způsoby ochrany, obsah a rozdíly}

\newpage
\section[Smlouvy a software\,--\,licenční smlouva, smlouva o~dílo a SLA\,--\,pojem, strany, obsah, forma a proces uzavírání]{Smlouvy a software\,--\,licenční smlouva, smlouva o~dí-lo a SLA\,--\,pojem, strany, obsah, forma a proces uzavírání}

\newpage
\section{Veřejné licence a software, free a open source software}

\newpage
\section{Autorskoprávní a trestněprávní prostředky ochrany software\,--\,účel, nároky, tresty a realizace}

\newpage