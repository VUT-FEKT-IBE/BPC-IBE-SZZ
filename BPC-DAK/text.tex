\section{Teorie informace: Statické a dynamické vlastnosti zdroje informace, zdrojové kódování, statické a dynamické vlastnosti přizpůsobeného zdroje.}

\newpage
\section{Systémy přenosu informace: Kapacita diskrétního kanálu bez a s poruchami, kapacita spojitého kanálu, Shannonův vztah a jeho popis, kapacita kanálu a přenosová rychlost, výkon a výkonová spektrální hustota, jednotky.}

\newpage
\section{Přenos dat: Základní pojmy. Modulační a přenosová rychlost, jednotky. Provozní charakteristiky datových spojů. Systém přenosu dat: Struktura, rozhraní, normalizace.}

\newpage
\section[Kódování snižující nadbytečnost: Prefixové kódy. Kraftova~nerovnost\,--\,McMillanova věta. Huffmanův kód. Příklady kódů pro snížení nadbytečnosti. Bezztrátová a ztrátová komprese dat.]{Kódování snižující nadbytečnost: Prefixové kódy.\newline Kraftova~nerovnost\,--\,McMillanova věta. Huffmanův kód. Příklady kódů pro snížení nadbytečnosti. Bezztrátová a ztrátová komprese dat.}

\newpage
\section{ Protichybové kódování: Vznik chyby a její modelování. Druhy chyb. Základní koncepce protichybového kódování. Třídění protichybových kódů. Hammingova vzdálenost a váha. Zabezpečovací schopnost kódu.}

\newpage
\section{Protichybové blokové kódy: Třídění blokových kódů, způsoby zadávání blokových kódů, zabezpečení, detekce a oprava chyb. Příklady protichybových blokových kódů. Cyklické kódy: Zadávání, zabezpečení, detekce a oprava chyb. Příklady protichybových cyklických kódů.}

\newpage
\section{Stromové\,kódy:\,Třídění,\,popis\,stro\,ového\,kódu grafy a~diagramy, Konvoluční kódy, popis, způsoby zadávání konvolučních kódů, dekódování konvolučních kódů, Viterbiho algoritmus. Turbo kódy: Metody modifikace kódů, sériově a paralelně zřetězené kódy, prokládání, násobné kódy a turbo kódy, dekódování turbo kódů.}

\newpage
\section{Protichybové kódové systémy (PKS): Třídění, popis PKS s FEC a PKS s ARQ, příklady ARQ metod a~jejich srovnání. Propustnost a efektivní informační rychlost.}

\newpage
\section{Modemy: Struktura a vysvětlení významu jednotlivých bloků, Měniče v základním a v přeloženém pásmu, Linkové kódy, Klíčování, Modulace s jednou a více nosnými, protichybové zabezpečení modulačního procesu, skrambler, ekvalizér, duplexní provoz.}

\newpage
\section{Základy šifrování: Popis kryptografického systému, pravidla kryptografie, základní šifrovací postupy, symetrické a asymetrické kryptosystémy, příklady.}