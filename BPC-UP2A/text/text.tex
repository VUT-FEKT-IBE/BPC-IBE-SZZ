\section{Veřejnoprávní ochrana autorského práva}

\newpage
\section[Elektronický podpis a~elektronická pečeť -- právní úprava a~druhy]{Elektronický podpis a~elektronická pečeť -- \newline právní úprava a druhy}

\newpage
\section{Datové schránky -- právní úprava a~praxe používání}

\newpage
\section{Provozní a~lokalizační údaje a~jejich využití v trestním řízení}

\newpage
\section[Právní úprava kybernetické bezpečnosti v ČR -- základní principy]{Právní úprava kybernetické bezpečnosti v~ČR --\newline základní principy}

\newpage
\section{Ochrana osobních údajů -- vymezení osob a~základní principy zpracování osobních údajů}

\href{https://eur-lex.europa.eu/legal-content/CS/TXT/HTML/?uri=CELEX:32016R0679}{Nařízení 2016/679 o~ochraně fyzických osob v~souvislosti se~zpracováním osobních údajů [\dots] (GDPR)} a~česká implementace
\href{https://www.zakonyprolidi.cz/cs/2019-110}{Zákon č.~110/2019 Sb., o~zpracování osobních údajů}.

GDPR vychází z~\href{https://eur-lex.europa.eu/legal-content/cs/TXT/HTML/?uri=CELEX:31995L0046}{Data Protection Directive} z~roku 1995.

Hlavními prvky jsou prevenční princip (široká aplikace úpravy a~úzká aplikace výjimek, limitace a~vázanost účelem zpracování, \enquote{co není dovoleno je zakázáno}) a~obecnost (dopadá na~všechny právní osoby od~živnostníků přes~nadnárodní společnosti po~státy). Dodržování se nekontroluje aktivně, ale v~případě vyšetřování musí být veškerá dokumentace v~pořádku.

\subsection*{Pojmy}

\begin{itemize}[noitemsep]
\item \textbf{Osobní údaje}: veškeré informace o identikované/identifikovatelné fyzické osobě, kterou lze přímo či~nepřímo identifikovat.
\item \textbf{Subjekt údajů}: fyzická osoba.
\item \textbf{Zpracování}: sběr, uchovávání, distribuce a~veškerá manipulace s~údaji.
\item \textbf{Správce}: určovatel zpracování.
\item \textbf{Zpracovatel}: vykonavatel zpracování (může jít o~stejnou osobu).
\item \textbf{Účel} a~\textbf{Prostředky} zpracování.
\end{itemize}

\subsection*{Důvody zpracování}

\begin{itemize}[noitemsep]
\item \textbf{Souhlas subjektu údajů}.
\item \textbf{Smluvní závazek}: nutnost k~plnění smlouvy.
\item \textbf{Právní povinnost}: vyplývající ze~zákona, např. držení kontaktních údajů při~obchodním styku.
\item \textbf{Ochrana životně důležitých zájmů subjektu nebo FO}.
\item \textbf{Veřejný zájem nebo výkon veřejné moci}.
\item \textbf{Oprávněný zájem} správce nebo třetí osoby: např. ochrana infrastruktury.
\end{itemize}

\subsection*{Hlavní body GDPR}

\textbf{Vztahuje se na~hodně kategorií dat.} Směřuje k~ochraně všech osobních údajů: \emph{\enquote{Identifikovatelnou FO lze přímo či~nepřímo identifikovat jménem, číslem, lokačními údaji, síťovým identifikátorem nebo jedním či~více prvky fyzické, fyziologické, genetické, psychické, ekonomické, kulturní nebo společenské identity.}}. Omezená \enquote{implementace}: předpokládá se doplnění místní státní legislativou. Osobním údajem je i~IP/MAC adresa, UID nebo jiný (pseudo)unikátní identifikátor, pomocí jeho samotného nebo spojením s~ostatními může dojít k~identifikaci.

\textbf{Extrateritorialita.} Platí pro~FO/PO se sídlem v~EU, osoby nabízející služby EU rezidentům nebo monitorující jejich chování. Vztahuje se tedy i na~subjekty mimo EU. Subjekty ze~zahraničí musí mít v~některém ze~států EU svého zástupce.

\textbf{Vztahuje se i~na~zpracovatele}. Jsou povinni informovat o~uschování dat (jaká, za~jakým účelem, kde jsou, jak jsou zabezpečená) a~ohlašovat porušení zabezpečení. Data mohou zpracovávat pouze na~pokyn správce, je vynucena kontrola bezpečnosti subdodavatelů, mlčenlivost, nápomoc při~výkonu práv subjektu či vymazání dat.

\textbf{Důraz na~odpovědnost správců}. \emph{Privacy by design}. \emph{Privacy by default}. Správci se nemusí registrovat, mohou uschovávat pouze minimum údajů vyplývající z~důvodů zpracování. Musí dodržovat bezpečnost opatření: pseudonymizace a~šifrování, CIA\footnote{Confidentiality, Integrity, Availability} triáda, zálohování, revize ochrany.

\textbf{Posílení práv subjektu údajů.} Práva na~přístup, opravu, smazání a~blokování. GDPR posiluje nutnost jednoznačného souhlasu, zahrnuje právo na~zapomnění či~ochranu proti profilování. \\
Souhlas musí být samostatný a~srozumitelný, prokazuje jej správce; pro~subjekty mladší 16 (13) let existují zvláštní pravidla. Informace o~zpracování musí obsahovat účely, kategorie údajů, příjemce, dobu uschování, musí informovat o~právu na~výmaz, o~existenci profilování nebo automatického rozhodování.

\textbf{Oznamování narušení bezpečnosti.} Musí proběhnout do~72 hodin. Informování subjektu pokud existuje vysoké riziko, úřadu při~jakémkoliv narušení: povaha, kategorie, počet subjektů a~údajů, důsledky, přijatá opatření.

\textbf{Předávání dat mimo EU.} Bylo velmi omezeno. Koncept odpovídající ochrany, vhodných záruk a~povolení.

\textbf{Pokuty}. Až~€20M nebo 4\,\% celosvětového obratu (dle toho co je vyšší). Auditování dozorovými úřady. \emph{One Stop Shop} (možnost kontaktovat místní úřad pro~ochranu osobních údajů, i~když je správce ve~státě jiném).

\newpage
\section{Povinnosti správce a~zpracovatele osobních údajů}
\label{question-7}

\emph{\enquote{Správce zavede vhodná technická a~organizační opatření, aby zajistil a~byl schopen doložit, že je zpracování prováděno v~souladu s~nařízením GDPR. Tato opatření musí být podle potřeby revidována a aktualizována.}}

\emph{\enquote{S~přihlédnutím ke~stavu techniky, nákladům na~provedení, povaze, rozsahu, kontextu a~účelům zpracování [\dots] zavede správce [\dots] vhodná technická a organizační opatření, jako je pseudonymizace, jejichž účelem je provádět zásady ochrany údajů, jako je minimalizace údajů [\dots].}}

Musí být zpracovávány pouze osobní údaje nezbytně nutné pro~konkrétní účel zpracování. Tyto údaje nesmí být standardně bez~zásahu člověka zpřístupněny neomezenému počtu fyzických osob.

\vspace*{1em}

V~okamžiku získání osobních údajů správce poskytne%
\footnote{%
	\href{https://eur-lex.europa.eu/legal-content/CS/TXT/HTML/?uri=CELEX:32016R0679\#d1e2243-1-1}{dle 2016/679/ES, čl. 13}%
}:

\begin{itemize}[noitemsep]
\item totožnost a~kontaktní údaje správce (a~jeho zástupce), případně kontaktní údje pověřence pro~ochranu osobních údajů
\item účely zpracování a~jejich právní základ
\item příjemce nebo kategorie příjemců osobních údajů
\item úmysl předat osobní údaje do~třetí země nebo mezinárodní organizaci
\item dobu, po~kterou budou osobní údaje uloženy, případně kritéria pro~stanovení takové doby
\item existenci práva požadovat přístup/\dots k~osobním údaům týkajícím se subjektu údajů
\item existenci práva odvolat souhlas, podat stížnost u~dozorového úřadu
\item jestli jde o~zákonný nebo smluvní požadavek, zda má subjekt možnost údaje neposkytnout a~důsledky neposkytnutí
\end{itemize}

\begin{center}
{\huge \dots} zde je třeba doplnit zbytek {\huge \dots}
% https://eur-lex.europa.eu/legal-content/CS/TXT/HTML/?uri=CELEX:32016R0679#d1e3011-1-1
\end{center}

\newpage
\section{Práva subjektu údajů ve~vztahu ke~správci a~zpracovateli}

\href{https://eur-lex.europa.eu/legal-content/CS/TXT/HTML/?uri=CELEX:32016R0679#d1e2150-1-1}{Nařízení 2016/679 (GDPR), kapitola III: Práva subjektu údajů}.

Správce poskytuje údaje stručně, transparentně, srozumitelně a~snadno přístupným způsobem. Informace poskytne bez zbytečného odkladu a~nejdéle do~jednoho měsíce od~obdržení žádosti (v~případě potřeby, složitosti a~vysokém počtu žádostí lze prodloužit o~dva měsíce). Veškeré úkony jsou bezplatné; pokud jsou žádosti nedůvodné nebo nepřiměřené (opakují se), může správce uložit poplatek zohledňující náklady, nebo může žádosti odmítnout vyhovět.

\subsection*{Informování}

Dle~článků 13 a~14 má právo být informován o~sběru osobních údajů (viz otázku~\ref{question-7}), dle~článku 15 má právo na~přístup ke~sbíraným údajům.

Dle~článku 16 má právo na~opravu nepřesných a na~doplnění neúplných osobních údajů.

\subsection*{Výmaz a~omezení}

Dle článku 17 má právo na~výmaz (\enquote{právo být zapomenut}), pokud jeho údaje již nejsou potřebné pro~deklarované účely, subjekt odvolává souhlas, vznese námitku proti~zpracování nebo jsou jeho údaje zpracovávány protiprávně. Toto právo se neuplatní, pokud je zpracování nezbytené pro~výkon práva na~svobodu projevu a~informace, pro~splnění právní povinnosti nebo pro~splnění úkolu provedeného ve~veřejném zájmu nebo při~výkonu veřejné moci nebo pro~určení, výkon nebo obhajobu právních nároků.

Dle článku 18 má právo na~omezení zpracování, pokud subjekt popírá přesnost osobních údajů (to na~dobu potřebnou k~ověření přesnosti správcem), zpracování je~protiprávní a~subjekt odmítá výmaz osobních údajů, správce údaje nepotřebuje, ale subjekt je požaduje pro výkon právních nároků.

\subsection*{Marketing, profilování a~automatické zpracování}

Dle článku 21 má právo vznést námitku proti~zpracování včetně profilování. Subjekt má vždy právo vznést námitku proti~zpracování pro~účely marketingu.

Subjekt má právo nebýt předmětem rozhodnutí založeného výhradně na~automatizovaném zpracování. Toto právo se nepoužije pokud to není nezbytné k~uzavření nebo plnění smlouvy, pokud je to povoleno právem Unie/členského státu nebo pokud je založeno na~výslovném souhlasu subjektu.

\newpage
\section{Informace veřejného sektoru -- pojem, česká a~evropská právní úprava}

\newpage
\section{Otevřená data -- pojem a~právní úprava}

Jde o~úplná, snadno dostupná, strojově čitelná data používající standardy s~volně dostupnou specifikací, která jsou zpřístupněna za~jasně daných podmínek užití s~minimem omezení, dostupná uživatelům při~vynaložení minima úsilí.

Zákon č.~106/1999 Sb., o~svobodném přístupu k~informacím, §3 (11): \emph{\enquote{Otevřenými daty se rozumí informace zveřejňované v~otevřeném a~strojově čitelném formátu, jejichž způsob ani účel následného využití není omezen a~jsou evidovány v~národním katalogu otevřených dat.}}

\begin{itemize}[noitemsep]
\item data jsou online
\item data jsou online ve~strukturované podobě
\item data jsou online ve~strukturované podobě v~otevřeném formátu
\item data jsou online ve~strukturované podobě v~otevřeném formátu a~mají vlastní IRI
\item data jsou online ve~strukturované podobě v~otevřeném formátu, mají vlastní IRI a~jsou přímo propojená s~dalšími datovými zdroji (linkování)
\end{itemize}

Povinná data jsou například jízdní řády, metadata registru smluv nebo příjemci dotací.

Mezi právními problémy jsou např. špatně nastavené smlouvy s~provozovateli dat (vendor lock-in -- \href{http://ictjudikatura.law.muni.cz/wiki/6_As_38/2015_-_51_-_%C5%BD%C3%A1dost_o_specifick%C3%BD_form%C3%A1t_informac%C3%AD_(CHAPS)}{spor CHAPS v.~Seznam o~jízdní řády}), smlouvy zastaralé nebo neexistující, práva duševního vlastnictví (autorské právo, právo pořizovatele databáze -- u~státních dat neexistují a~nic licencovat není třeba), anonymizace (dle GDPR).

Např.~\href{https://mapaexekuci.cz}{mapa exekucí}, \href{https://prazdnedomy.cz}{prázdné domy}, \href{https://hlidacstatu.cz}{Hlídač státu}, \href{https://data.brno.cz}{data.Brno}.
