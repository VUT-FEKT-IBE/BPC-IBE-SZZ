\section{Veřejnoprávní ochrana autorského práva}

\newpage
\section[Elektronický podpis a~elektronická pečeť -- právní úprava a~druhy]{Elektronický podpis a~elektronická pečeť -- \newline právní úprava a druhy}

\newpage
\section{Datové schránky -- právní úprava a~praxe používání}

\newpage
\section{Provozní a~lokalizační údaje a~jejich využití v trestním řízení}

\newpage
\section[Právní úprava kybernetické bezpečnosti v ČR -- základní principy]{Právní úprava kybernetické bezpečnosti v~ČR --\newline základní principy}

\newpage
\section{Ochrana osobních údajů -- vymezení osob a~základní principy zpracování osobních údajů}

\href{https://eur-lex.europa.eu/legal-content/CS/TXT/HTML/?uri=CELEX:32016R0679}{Nařízení 2016/679 o~ochraně fyzických osob v~souvislosti se~zpracováním osobních údajů [\dots] (GDPR)} a~česká implementace
\href{https://www.zakonyprolidi.cz/cs/2019-110}{Zákon č.~110/2019 Sb., o~zpracování osobních údajů}.

GDPR vychází z~\href{https://eur-lex.europa.eu/legal-content/cs/TXT/HTML/?uri=CELEX:31995L0046}{Data Protection Directive} z~roku 1995.

Hlavními prvky jsou prevenční princip (široká aplikace úpravy a~úzká aplikace výjimek, limitace a~vázanost účelem zpracování, \enquote{co není dovoleno je zakázáno}) a~obecnost (dopadá na~všechny právní osoby od~živnostníků přes~nadnárodní společnosti po~státy). Dodržování se nekontroluje aktivně, ale v~případě vyšetřování musí být veškerá dokumentace v~pořádku.

\subsection*{Pojmy}

\begin{itemize}[noitemsep]
\item \textbf{Osobní údaje}: veškeré informace o identikované/identifikovatelné fyzické osobě, kterou lze přímo či~nepřímo identifikovat.
\item \textbf{Subjekt údajů}: fyzická osoba.
\item \textbf{Zpracování}: sběr, uchovávání, distribuce a~veškerá manipulace s~údaji.
\item \textbf{Správce}: určovatel zpracování.
\item \textbf{Zpracovatel}: vykonavatel zpracování (může jít o~stejnou osobu).
\item \textbf{Účel} a~\textbf{Prostředky} zpracování.
\end{itemize}

\subsection*{Důvody zpracování}

\begin{itemize}[noitemsep]
\item \textbf{Souhlas subjektu údajů}.
\item \textbf{Smluvní závazek}: nutnost k~plnění smlouvy.
\item \textbf{Právní povinnost}: vyplývající ze~zákona, např. držení kontaktních údajů při~obchodním styku.
\item \textbf{Ochrana životně důležitých zájmů subjektu nebo FO}.
\item \textbf{Veřejný zájem nebo výkon veřejné moci}.
\item \textbf{Oprávněný zájem} správce nebo třetí osoby: např. ochrana infrastruktury.
\end{itemize}

\subsection*{Hlavní body GDPR}

\textbf{Vztahuje se na~hodně kategorií dat.} Směřuje k~ochraně všech osobních údajů: \emph{\enquote{Identifikovatelnou FO lze přímo či~nepřímo identifikovat jménem, číslem, lokačními údaji, síťovým identifikátorem nebo jedním či~více prvky fyzické, fyziologické, genetické, psychické, ekonomické, kulturní nebo společenské identity.}}. Omezená \enquote{implementace}: předpokládá se doplnění místní státní legislativou. Osobním údajem je i~IP/MAC adresa, UID nebo jiný (pseudo)unikátní identifikátor, pomocí jeho samotného nebo spojením s~ostatními může dojít k~identifikaci.

\textbf{Extrateritorialita.} Platí pro~FO/PO se sídlem v~EU, osoby nabízející služby EU rezidentům nebo monitorující jejich chování. Vztahuje se tedy i na~subjekty mimo EU. Subjekty ze~zahraničí musí mít v~některém ze~států EU svého zástupce.

\textbf{Vztahuje se i~na~zpracovatele}. Jsou povinni informovat o~uschování dat (jaká, za~jakým účelem, kde jsou, jak jsou zabezpečená) a~ohlašovat porušení zabezpečení. Data mohou zpracovávat pouze na~pokyn správce, je vynucena kontrola bezpečnosti subdodavatelů, mlčenlivost, nápomoc při~výkonu práv subjektu či vymazání dat.

\textbf{Důraz na~odpovědnost správců}. \emph{Privacy by design}. \emph{Privacy by default}. Správci se nemusí registrovat, mohou uschovávat pouze minimum údajů vyplývající z~důvodů zpracování. Musí dodržovat bezpečnost opatření: pseudonymizace a~šifrování, CIA\footnote{Confidentiality, Integrity, Availability} triáda, zálohování, revize ochrany.

\textbf{Posílení práv subjektu údajů.} Práva na~přístup, opravu, smazání a~blokování. GDPR posiluje nutnost jednoznačného souhlasu, zahrnuje právo na~zapomnění či~ochranu proti profilování. \\
Souhlas musí být samostatný a~srozumitelný, prokazuje jej správce; pro~subjekty mladší 16 (13) let existují zvláštní pravidla. Informace o~zpracování musí obsahovat účely, kategorie údajů, příjemce, dobu uschování, musí informovat o~právu na~výmaz, o~existenci profilování nebo automatického rozhodování.

\textbf{Oznamování narušení bezpečnosti.} Musí proběhnout do~72 hodin. Informování subjektu pokud existuje vysoké riziko, úřadu při~jakémkoliv narušení: povaha, kategorie, počet subjektů a~údajů, důsledky, přijatá opatření.

\textbf{Předávání dat mimo EU.} Bylo velmi omezeno. Koncept odpovídající ochrany, vhodných záruk a~povolení.

\textbf{Pokuty}. Až~€20M nebo 4\,\% celosvětového obratu (dle toho co je vyšší). Auditování dozorovými úřady. \emph{One Stop Shop} (možnost kontaktovat místní úřad pro~ochranu osobních údajů, i~když je správce ve~státě jiném).

\newpage
\section{Povinnosti správce a~zpracovatele osobních údajů}

\newpage
\section{Práva subjektu údajů ve~vztahu ke~správci a~zpracovateli}

\newpage
\section{Informace veřejného sektoru -- pojem, česká a~evropská právní úprava}

\newpage
\section{Otevřená data -- pojem a~právní úprava}
