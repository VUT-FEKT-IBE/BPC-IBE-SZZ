\section{Veřejnoprávní ochrana autorského práva}

Ochrana práva se dělí na soukromoprávní a~veřejnoprávní. Nejčastěji je narušení autorského řešeno v~občanskoprávním sporu. Spadá pod ochranu duševního vlastnictví.

Veřejnoprávní rovina se dělí na~správněprávní, trestněprávní a~ústavněprávní.

Odpovědnost se dělí na:
\vspace{-0,2cm}
\begin{itemize}[noitemsep]
    \item Občanskoprávní, která spadá do soukroměprávní roviny.
    \item Přestupková, spadá do veřejnoprávní roviny (správněprávní).
    \item Trestní, spadá do veřejnoprávní roviny (trestněprávní).
\end{itemize}

Výše škody:
\vspace{-0,1cm}
\begin{itemize}[noitemsep]
    \item Nikoliv nepatrná škoda dosahující částky 10\,000\,Kč.
    \item Nikoliv malou škodu dosahující částky 50\,000\,Kč.
    \item Škodou dosahující částky 100\,000\,Kč.
    \item Značnou škodou dosahující částky 1\,000\,000\,Kč.
    \item Škodou velkého rozsahu dosahující částky 10\,000\,000\,Kč.
\end{itemize}

Minimální výše škody jsou uvedeny, jelikož se dle nich odvyjí jaká sazba bude udělena v~případě trestního řízení a~ke specifikaci, kdy se ještě jedná o~přestupek a~ne trestný čin.

Do~\textbf{správněprávní} spadají přestupky, řeší je obce s~rozšířenou působností. Přestupku se dle 
\href{https://www.zakonyprolidi.cz/cs/2000-121#p105a-1-a}{§105 AZ} dopustí FO, PO nebo podnikající FO, která neoprávněně užije autorské dílo, umělecký výkon, zvukový či zvukově obrazový záznam, rozhlasové nebo televizní vysílání nebo databázi. U~FO lze uložit sankci od 100\,000\,Kč do 150\,000\,Kč dle spáchaného přestupku. U~PO a~podnikajících FO lze uložit sankci od 50\,000\,Kč do 500\,000\,Kč dle spáchaného přestupku.
Přestupky na~rozdíl od~trestných činů mohou trestat větší okruh jednání (nedbalost; neúmyslnost při~povinnosti vědět). Trestný čin může být spáchán pouze a~jedině úmyslně. 

\textbf{Trestněprávní právo} z~pohledu porušení autorského práva je popsáno v~\href{https://www.zakonyprolidi.cz/cs/2009-40#p270}{§270~TZ}. Při~porušení autorského práva není potřeba způsobit minimální škodu, na~rozdíl od~skutkových podstat, ale často je k ní přihlíženo v případě určování trestu. Účelem trestněprávního práva je potrestat osobu, která se dopustila porušení práv a~žádat náhradu způsobené škody.

Sazba za~porušení může být dle \href{https://www.zakonyprolidi.cz/cs/2009-40#p270}{§270~TZ} odnětí svobody až na~2 roky, zákaz činnosti nebo propadnutí věci. Dále pokud by by porušení autorského zákona vykazovalo znaky obchodní činnosti nebo jiného podnikaní (je-li porušení ve~značném rozsahu) mění se sazba na~6~měsíců až 5~let odnětí svobody, peněžitý trest nebo propadnutí věci. Pokud by pachatel porušil autorská práva ve~velkém rozsahu, může být potrestán na~3--8 roků odnětí svobody.

Osoba, které bylo narušeno autorské právo, podává trestní oznámení na~policii, poté už \uv{nic neřeší}. Probíhá \textbf{přípravné řízení}, ve~kterém policie šetří co se stalo a~shromažďují se důkazy. V~této fázi se také určí jestli došlo vůbec k~trestnému činu. Pokud se jedná trestný čin, podává se \textbf{obžaloba}. Po~podání obžaloby je vyrozuměn obviněný, obhájce a~poškozený. Potom probíhá soudní řízení, kdy je rozhodnutu o~nevině nebo vině obžalovaného. V~případě autorské práva se v~průběhu musí vyřešit do~jakého autorského práva bylo zasaženo a~k~uznání viny musí být odůvodněno přesným paragrafem (blanketní skutková podstata).

\subsection{Celní správa}

Celní správa se stará o~boj proti padělání a~pirátství. Celní orgány jsou oprávněny zasáhnout proti zboží a~umožnit držitelům práv duševního vlastnictví jejich ochranu. Zboží porušující právo duševního vlastnictví bude zničeno, pokud nebylo bezúplatně převedeno k~humanitárním účelům nebo pokud není použito v~rámci působnosti orgánů celní správy.

\textbf{Padělek} je zboží včetně jeho obalu na němž jsou bez souhlasu majitele ochranné známky umístěno označení stejné nebo zaměnitelné s~originální ochranou známkou a~veškeré věci nesoucí stejné nebo obdobné označení. 

\textbf{Nedovolená napodobenina} je zboží včetně obalu, které bylo přímo nebo nepřímo bez souhlasu majitele nebo spolumajitele patentu, majitele autorského práva a~další, které je kopií nebo v~němž je kopie zahrnuta. Do toho spadá i~forma nebo raznice určená nebo upravená k~výrobě nedovolené napodobeniny.

Zjednodušeně padělek se snaží vydávat za originál zatímco nedovolená napodobenina je kopií originálu. Kopie filmu, nebo softwaru by měla být označena jako nedovolená napodobenina (nejsem si 100\% jistý).






\clearpage
\section[Elektronický podpis a~elektronická pečeť -- právní úprava a~druhy]{Elektronický podpis a~elektronická pečeť -- \newline právní úprava a druhy}

\vspace{-0,4cm}
\subsection{eIDAS}
\vspace{-0,4cm}

Cílem \href{https://eur-lex.europa.eu/legal-content/CS/TXT/?uri=CELEX\%3A32014R0910}{nařízení Evropského parlamentu a~Rady č.\,910/2014} eIDAS (\textbf{e}lectronic \textbf{id}entification \textbf{a}nd \textbf{s}ervices) je vytvoření jednotného rámce pro elektronickou identifikaci jednotlivých subjektů. Dalším důležitým cílem je zvýšení důvěrnosti elektronických transakcí na vnitřním trhu. V~České republice toto nařízení je adaptováno v podobě \href{https://www.zakonyprolidi.cz/cs/2016-297}{zákona č.\,297/2016}.

\textbf{Identifikace}\,--\,postup používaní osobních údajů, které jedinečně identifikují určitou osobu.

\textbf{Autentizace}\,--\,postup, který umožňuje potvrdit identifikaci nebo původ a~integritu dat v~elektronické podobě.

\vspace{-0,4cm}
\subsection{Elektronický podpis}
\vspace{-0,4cm}

Elektronický podpis dle eIDAS jsou data v~elektronické podobě, která jsou připojena k jiným datům v~elektronické podobě nebo jsou s~nimi logicky spojena a~která podepisující osoba používá k podepsání. Zaručený elektronický podpis umožňuje jednoznačnou identifikaci podepisující osoby. 

Elektronickému podpisu nesmějí být upírány právní účinky a~nesmí být odmítám v~soudním sporu z~důvodu, že má elektronickou podobu nebo že nesplňuje požadavky na kvalifikovaný elektronický podpis.

Elektronické podpisy se dělí na:
\begin{itemize}[noitemsep]
    \item Elektronický podpis (prostý)
    \item Zaručený elektronický podpis
    \item Uznávaný elektronický podpis
    \item Kvalifikovaný elektronický podpis
\end{itemize}

\textbf{Elektronický podpis (prostý)} je základní a~nejjednodušší typ elektronického podpisu. Sem se řadí napsání jména a~příjmení na konci dokumentu nebo emailu, vzor podpisu vytvořený elektronickou tužkou (to co bývá v bankách) a~další. Nejčastěji se využije mezi dvěma soukromoprávními subjekty, kdy nahrazuje obyčejný podpis.

\textbf{Zaručený elektronický podpis} má dle nařízení eIDAS splňovat:
\begin{itemize}[noitemsep]
    \item Je jednoznačně spojen s podepisující osobou.
    \item Umožňuje identifikaci podepisující osoby.
    \item Je vytvořen pomocí dat pro vytváření elektronických podpisů, která podepisující osoba může s~vysokou úrovní důvěry použít pod svou výhradní kontrolou.
    \item Je k~datům, která jsou tímto podpisem podepsaná připojen takovým způsobem, že je možné zjistit jakoukoliv následnou změnu dat.
\end{itemize}

Nevýhodou tohoto podpisu, i~když se zdá být bezpečný, je že podepisující osoba může vytvořit podpis jakékoliv osoby. Tento typ podpisu nezaručuje, kdo je autorem podpisu. 

\textbf{Uznávaný elektronický podpis} není zmíněn v~nařízení eIDAS, ale vychází z~české adaptace nařízení. Jedná se o~elektronický podpis založený na kvalifikovaném cerfitikátu pro elektronický podpis nebo o~kvalifikovaný elektronický podpis. Na rozdíl od zaručeného elektronického podpisu musí být potvrzen certifikační autoritou. Toto zaručuje spojitost s~osobou již byl vystaven. Pro komunikaci s~veřejnoprávním subjektem elektronicky je nutné použít tento typ elektronického podpisu nebo vyšší typ. Pokud má osoba zřízenou datovou schránku tak, elektronického podpisu nemusí využívat.

\textbf{Kvalifikovaný elektronický podpis} je dle nařízení eIDAS nejvyšší formou elektronického podpisu. V~nařízení je definován jako zaručený elektronický podpis, který je vytvořen kvalifikovaným prostředkem pro vytváření elektronických podpisů a~který je založen na kvalifikovaném certifikátu pro elektronické podpisy. Na rozdíl od uznávaného elektronického je při vytváření nutné použít kvalifikovaný prostředek pro vytváření elektronických podpisů, který je na samostatném nosiči a~je nepřenositelný na jiný nosič (USB token, čipová karta eID). Toto má výhodu, že nelze odcizit identitu bez toho aniž by byl odcizen nosič samotný.

\vspace{-0,4cm}
\subsubsection{Poskytovatelé služeb vytvářející důvěru}
\vspace{-0,4cm}

\textbf{Poskytovatel služeb vytvářejících důvěru} je FO nebo PO, která poskytuje jednu či více služeb vytvářejících důvěru buď jako kvalifikovaný, nebo jako nekvalifikovaný poskytovatel služeb vytvářejících důvěru.

\textbf{Kvalifikovaný poskytovatel služeb vytvářejících důvěru} je poskytovatel služeb vytvářejících důvěru, který poskytuje jednu či více kvalifikovaných služeb vytvářejících důvěru a~kterému orgán dohledu udělil status kvalifikovaného poskytovatele.

V České republice vydávají kvalifikované elektronické podpisy společnosti:
\begin{itemize}[noitemsep]
    \item První certifikační autorita
    \item Česká pošta
    \item eIdentity
\end{itemize}

\vspace{-0,4cm}
\subsection{Elektronická pečeť}
\vspace{-0,4cm}

Jde o~data v elektronické podobě, připojená k~jiným datům, která chceme zapečetit, v~elektronické podobě s~cílem zaručit jejich pravost a~neměnnost.

Existují 4 druhy elektronických pečetí:
\begin{itemize}[noitemsep]
    \item Prostá elektronická pečeť.
    \item Zaručená elektronická pečeť.
    \item Zaručená elektronická pečeť s certifikátem.
    \item Kvalifikovaná elektronická pečeť.
\end{itemize}

Z~těchto typů pečetí je pouze v~EU povinně uznávána kvalifikovaná elektronická pečeť a~je vytvářena podobně jako kvalifikovaný elektronický podpis. Elektronická pečeť by měla sloužit že elektronický dokument byl vydán určitou právnickou osobou. Pečeť tímto není spojená s~určitou osobou na rozdíl od elektronického podpisu (rozdíl PO a~FO).
Elektronickou pečeť lze krom podepisovaní dokumentů také využít k~podpisu elektronického časového razítka.

\vspace{-0,4cm}
\subsection{Elektronická časové razítko} Dle nařízení eIDAS to jsou data v~elektronické podobě, která spojují jiná data v~elektronické podobě s~určitým okamžikem a~prokazují, že tato jiná data existovala v~daném okamžiku.







\clearpage
\section{Datové schránky -- právní úprava a~praxe používání}

\clearpage
\section{Provozní a~lokalizační údaje a~jejich využití v trestním řízení}

\clearpage
\section[Právní úprava kybernetické bezpečnosti v ČR -- základní principy]{Právní úprava kybernetické bezpečnosti v~ČR --\newline základní principy}

Kybernetická bezpečnost je definována \href{https://www.zakonyprolidi.cz/cs/2014-181}{zákonem č. 181/2014 o kybernetické bezpečnosti}. Kde prvky kritické infrastruktury jsou definovány v \href{https://www.zakonyprolidi.cz/cs/2010-432}{nařízením vlády č. 432/2010}.

Kybernetická bezpečnost je aktivita, která se zabývá ochranou počítačových systémů před poškozením a narušením provozu hardwaru, softwaru nebo informací(souhrn prostředků směřujících k zajištění ochrany kybernetického prostoru). Jejím primárním cílem je zajistit CIA (Confidentiality\,--\,důvěrnost, integrity\,--\,integrita a availability\,--\,dostupnost) triádu spravovaných sítích. Hlavním smyslem je pak ochrana prostředí k realizaci informačních práv člověka.

\textbf{Kybernetický prostor} je digitální prostředí umožnující vznik, zpracovaní a výměnu informací tvořené informačními systémy, službami a sítěmi elektronických komunikací. Definovaný v  \href{https://www.zakonyprolidi.cz/cs/2014-181#p2-1-a}{§2 odst.\,1 písm.\,a) zákona 181/2014}.

Povinnost zajišťovat kybernetickou bezpečnost mají:
\begin{itemize}[noitemsep]
    \item Soukromé subjekty.
    \begin{itemize}[noitemsep]
        \item Poskytovatelé služeb elektronických komunikací \href{https://www.zakonyprolidi.cz/cs/2014-181#p3-1-a}{§3 odst.\,1 písm.\,a) zákona\\ 181/2014}
        \item Osoba zajišťující významnou síť \href{https://www.zakonyprolidi.cz/cs/2014-181#p3-1-b}{§3 odst.\,1 písm.\,b) zákona 181/2014}
    \end{itemize}
    \item Soukromé a veřejné subjekty.
    \begin{itemize}[noitemsep]
        \item Správce informačních systému/komunikačních systémů kritické informační infrastruktury \href{https://www.zakonyprolidi.cz/cs/2014-181#p3-1-c}{§3 odst.\,1 písm.\,c) a písm.\,d) zákona 181/2014}
    \end{itemize}
    \item Veřejné subjekty
    \begin{itemize}[noitemsep]
        \item Správce významného informačního systému \href{https://www.zakonyprolidi.cz/cs/2014-181#p3-1-e}{§3 odst.\,1 písm.\,e) zákona 181/2014}
    \end{itemize}
    \item Síťová informační služba (NIS)
    \begin{itemize}[noitemsep]
        \item Poskytovatel základních služeb \href{https://www.zakonyprolidi.cz/cs/2014-181#p3-1-f}{§3 odst.\,1 písm.\,f) zákona 181/2014}
        \item Poskytovatel digitální služby \href{https://www.zakonyprolidi.cz/cs/2014-181#p3-1-h}{§3 odst.\,1 písm.\,h) zákona 181/2014}
    \end{itemize}
\end{itemize}

\noindent\textbf{Kritickou infrastrukturou} se rozumí prvek nebo systém prvků kritické infrastruktury, jehož narušení by mělo závažný na bezpečnost a chod státu. Můžou do toho spadat některá z odvětví jako energetika, doprava, zdravotnictví, digitální infrastruktura a další spadající do kategorií definovaných v \href{https://www.zakonyprolidi.cz/cs/2014-181#p2-1-i}{§2 odst.\,1 písm.\,i)  zákona 181/2014}.

\noindent\textbf{Kritickou informační infrastrukturou} jsou kybernetické složky kritické infrastruktury. Například ovládací systémy (SCADA) nebo obecně cyber-physical systemy (CPS).

\noindent\textbf{Prvkem kritické infrastruktury} je například elektrárna nebo v případě poskytovatelů elektronických komunikačních služeb jejich síť.

\noindent\textbf{Provozovatelem kritické služby} je vlastník daného prvku kritické infrastruktury.

Provozovatel má určité povinnosti, které se odvyjí od toho co přesně provozuje. Těmito povinnostmi jsou:
\begin{itemize}[noitemsep]
    \item Obecné povinnosti (kontaktní údaje, organizační a technická bezpečnostní opatření), které se vztahují na většinu provozovatelů a jsou preventivního charakteru. Je to nejnižší stupeň přijímaných opatření.
    \item  Operativní povinnosti (hlášení incidentů, varování, reaktivní a ochranná protiopatření) slouží pro případ spolupráce při nějakém stavu ohrožení/incidentu. Zde už musí být nastaveny nějaké mechanizmy pro detekci incidentů.
    \item Stav kybernetického nebezpečí\,--\,koordinace při nějakém větším útoku. Při vyhlášení se rozšíří pravomoc ve vztahu k povinným osobám. 
    \item Požadavky na dodavatele
    \item Certifikace
\end{itemize}
NÚKIB může vydat opatření:
\begin{itemize}[noitemsep]
    \item Varovaní \href{https://www.zakonyprolidi.cz/cs/2014-181#p12}{§12 zákona 181/2014}. Jeli nalezena hrozba/zranitelnost/riziko v oblasti kyberbezpečnosti. 
    \item Reaktivní opatření \href{https://www.zakonyprolidi.cz/cs/2014-181#p13}{§13 zákona 181/2014}. Jestliže se něco děje tak vydává opatření aby povinné subjekty reagovali určitým způsobem.
    \item Ochranné opatření \href{https://www.zakonyprolidi.cz/cs/2014-181#p14}{§14 zákona 181/2014}. Vydává se opatření až po proběhlém bezpečnostním incidentu, aby se povinné subjekty zabezpečili.
\end{itemize}

\textbf{Stav kybernetického nebezpečí}\,--\,je definován hlavně z důvodu jelikož ne všechny subjekty mají povinnost provádět reaktivní a ochranná opatření. Proto při nějakém rozsáhlejším útoku, tak stát potřebuje efektivně reagovat tak tento stav rozšiřuje povinnosti na ostatní subjekty, na které vyšší typy opatřené normálně neplatí. Tento stav vyhlašuje ředitel NÚKIB  a musí se zveřejnit že byl vyhlášen. Maximální délka vyhlášení je 7 dnů s prodloužením až na 30 dní.

Bezpečnost pro organizace zajišťují CSIRT týmy, které se zabezpečují svoji izolovanou infrastrukturu. Na úrovní státu bezpečnost zajišťují v ČR národní a vládní CERT týmy, které koordinují postupy provozovatelů sítí a systémů. Na mezinárodní úrovni jsou to různé mezinárodní organizace a dobrovolné spolky, které především umožňují sdílení informací (například ENISA).

\textbf{Vládní CERT} je definován v \href{https://www.zakonyprolidi.cz/cs/2014-181#p20}{§20 zákona 181/2014} spadá sem NCKB (Národní centrum kybernetické bezpečnosti)/GovCERT.CZ, který je částí NÚKIBU. Přijímá prohlášení o kybernetických a bezpečnostních incidentech. Vyhodnocuje údaje o kybernetických bezpečnostních událostech. Poskytuje součinnost při výskytu incidentu/události. Přijímá údaje od provozovatele národního CERTu a vyhodnocuje je. Spolupracuje s CSIRT týmy jiných států.

\textbf{Národní CERT} je definován v \href{https://www.zakonyprolidi.cz/cs/2014-181#p17}{§17 zákona 181/2014} a v ČŘ ho provozuje CZ.NIC na základně veřejnoprávní smlouvy. Přijímá oznámení kontaktních údajů a hlášení o kybernetických bezpečnostních incidentech. Vyhodnocuje kybernetické bezpečnostní incidenty u povinných subjektů. Působí jako kontaktní místo a podobně jako vládní spolupracuje s CSIRT týmy.







\clearpage
\section{Ochrana osobních údajů -- vymezení osob a~základní principy zpracování osobních údajů}

\href{https://eur-lex.europa.eu/legal-content/CS/TXT/HTML/?uri=CELEX:32016R0679}{Nařízení 2016/679 o~ochraně fyzických osob v~souvislosti se~zpracováním osobních údajů [\dots] (GDPR)} a~česká implementace
\href{https://www.zakonyprolidi.cz/cs/2019-110}{Zákon č.~110/2019 Sb., o~zpracování osobních údajů}.

GDPR vychází z~\href{https://eur-lex.europa.eu/legal-content/cs/TXT/HTML/?uri=CELEX:31995L0046}{Data Protection Directive} z~roku 1995.

Hlavními prvky jsou prevenční princip (široká aplikace úpravy a~úzká aplikace výjimek, limitace a~vázanost účelem zpracování, \enquote{co není dovoleno je zakázáno}) a~obecnost (dopadá na~všechny právní osoby od~živnostníků přes~nadnárodní společnosti po~státy). Dodržování se nekontroluje aktivně, ale v~případě vyšetřování musí být veškerá dokumentace v~pořádku.

\subsection*{Pojmy}

\begin{itemize}[noitemsep]
\item \textbf{Osobní údaje}: veškeré informace o identikované/identifikovatelné fyzické osobě, kterou lze přímo či~nepřímo identifikovat.
\item \textbf{Subjekt údajů}: fyzická osoba.
\item \textbf{Zpracování}: sběr, uchovávání, distribuce a~veškerá manipulace s~údaji.
\item \textbf{Správce}: určovatel zpracování.
\item \textbf{Zpracovatel}: vykonavatel zpracování (může jít o~stejnou osobu).
\item \textbf{Účel} a~\textbf{Prostředky} zpracování.
\end{itemize}

\subsection*{Důvody zpracování}

\begin{itemize}[noitemsep]
\item \textbf{Souhlas subjektu údajů}.
\item \textbf{Smluvní závazek}: nutnost k~plnění smlouvy.
\item \textbf{Právní povinnost}: vyplývající ze~zákona, např. držení kontaktních údajů při~obchodním styku.
\item \textbf{Ochrana životně důležitých zájmů subjektu nebo FO}.
\item \textbf{Veřejný zájem nebo výkon veřejné moci}.
\item \textbf{Oprávněný zájem} správce nebo třetí osoby: např. ochrana infrastruktury.
\end{itemize}

\subsection*{Hlavní body GDPR}

\textbf{Vztahuje se na~hodně kategorií dat.} Směřuje k~ochraně všech osobních údajů: \emph{\enquote{Identifikovatelnou FO lze přímo či~nepřímo identifikovat jménem, číslem, lokačními údaji, síťovým identifikátorem nebo jedním či~více prvky fyzické, fyziologické, genetické, psychické, ekonomické, kulturní nebo společenské identity.}}. Omezená \enquote{implementace}: předpokládá se doplnění místní státní legislativou. Osobním údajem je i~IP/MAC adresa, UID nebo jiný (pseudo)unikátní identifikátor, pomocí jeho samotného nebo spojením s~ostatními může dojít k~identifikaci.

\textbf{Extrateritorialita.} Platí pro~FO/PO se sídlem v~EU, osoby nabízející služby EU rezidentům nebo monitorující jejich chování. Vztahuje se tedy i na~subjekty mimo EU. Subjekty ze~zahraničí musí mít v~některém ze~států EU svého zástupce.

\textbf{Vztahuje se i~na~zpracovatele}. Jsou povinni informovat o~uschování dat (jaká, za~jakým účelem, kde jsou, jak jsou zabezpečená) a~ohlašovat porušení zabezpečení. Data mohou zpracovávat pouze na~pokyn správce, je vynucena kontrola bezpečnosti subdodavatelů, mlčenlivost, nápomoc při~výkonu práv subjektu či vymazání dat.

\textbf{Důraz na~odpovědnost správců}. \emph{Privacy by design}. \emph{Privacy by default}. Správci se nemusí registrovat, mohou uschovávat pouze minimum údajů vyplývající z~důvodů zpracování. Musí dodržovat bezpečnost opatření: pseudonymizace a~šifrování, CIA\footnote{Confidentiality, Integrity, Availability} triáda, zálohování, revize ochrany.

\textbf{Posílení práv subjektu údajů.} Práva na~přístup, opravu, smazání a~blokování. GDPR posiluje nutnost jednoznačného souhlasu, zahrnuje právo na~zapomnění či~ochranu proti profilování. \\
Souhlas musí být samostatný a~srozumitelný, prokazuje jej správce; pro~subjekty mladší 16 (13) let existují zvláštní pravidla. Informace o~zpracování musí obsahovat účely, kategorie údajů, příjemce, dobu uschování, musí informovat o~právu na~výmaz, o~existenci profilování nebo automatického rozhodování.

\textbf{Oznamování narušení bezpečnosti.} Musí proběhnout do~72 hodin. Informování subjektu pokud existuje vysoké riziko, úřadu při~jakémkoliv narušení: povaha, kategorie, počet subjektů a~údajů, důsledky, přijatá opatření.

\textbf{Předávání dat mimo EU.} Bylo velmi omezeno. Koncept odpovídající ochrany, vhodných záruk a~povolení.

\textbf{Pokuty}. Až~€20M nebo 4\,\% celosvětového obratu (dle toho co je vyšší). Auditování dozorovými úřady. \emph{One Stop Shop} (možnost kontaktovat místní úřad pro~ochranu osobních údajů, i~když je správce ve~státě jiném).

\clearpage
\section{Povinnosti správce a~zpracovatele osobních údajů}
\label{question-7}

\emph{\enquote{Správce zavede vhodná technická a~organizační opatření, aby zajistil a~byl schopen doložit, že je zpracování prováděno v~souladu s~nařízením GDPR. Tato opatření musí být podle potřeby revidována a aktualizována.}}

\emph{\enquote{S~přihlédnutím ke~stavu techniky, nákladům na~provedení, povaze, rozsahu, kontextu a~účelům zpracování [\dots] zavede správce [\dots] vhodná technická a organizační opatření, jako je pseudonymizace, jejichž účelem je provádět zásady ochrany údajů, jako je minimalizace údajů [\dots].}}

Musí být zpracovávány pouze osobní údaje nezbytně nutné pro~konkrétní účel zpracování. Tyto údaje nesmí být standardně bez~zásahu člověka zpřístupněny neomezenému počtu fyzických osob.

\vspace*{1em}

V~okamžiku získání osobních údajů správce poskytne%
\footnote{%
	\href{https://eur-lex.europa.eu/legal-content/CS/TXT/HTML/?uri=CELEX:32016R0679\#d1e2243-1-1}{dle 2016/679/ES, čl. 13}%
}:

\begin{itemize}[noitemsep]
\item totožnost a~kontaktní údaje správce (a~jeho zástupce), případně kontaktní údje pověřence pro~ochranu osobních údajů
\item účely zpracování a~jejich právní základ
\item příjemce nebo kategorie příjemců osobních údajů
\item úmysl předat osobní údaje do~třetí země nebo mezinárodní organizaci
\item dobu, po~kterou budou osobní údaje uloženy, případně kritéria pro~stanovení takové doby
\item existenci práva požadovat přístup/\dots k~osobním údaům týkajícím se subjektu údajů
\item existenci práva odvolat souhlas, podat stížnost u~dozorového úřadu
\item jestli jde o~zákonný nebo smluvní požadavek, zda má subjekt možnost údaje neposkytnout a~důsledky neposkytnutí
\end{itemize}

\begin{center}
{\huge \dots} zde je třeba doplnit zbytek {\huge \dots}
% https://eur-lex.europa.eu/legal-content/CS/TXT/HTML/?uri=CELEX:32016R0679#d1e3011-1-1
\end{center}

\clearpage
\section{Práva subjektu údajů ve~vztahu ke~správci a~zpracovateli}

\href{https://eur-lex.europa.eu/legal-content/CS/TXT/HTML/?uri=CELEX:32016R0679#d1e2150-1-1}{Nařízení 2016/679 (GDPR), kapitola III: Práva subjektu údajů}.

Správce poskytuje údaje stručně, transparentně, srozumitelně a~snadno přístupným způsobem. Informace poskytne bez zbytečného odkladu a~nejdéle do~jednoho měsíce od~obdržení žádosti (v~případě potřeby, složitosti a~vysokém počtu žádostí lze prodloužit o~dva měsíce). Veškeré úkony jsou bezplatné; pokud jsou žádosti nedůvodné nebo nepřiměřené (opakují se), může správce uložit poplatek zohledňující náklady, nebo může žádosti odmítnout vyhovět.

\subsection*{Informování}

Dle~článků 13 a~14 má právo být informován o~sběru osobních údajů (viz otázku~\ref{question-7}), dle~článku 15 má právo na~přístup ke~sbíraným údajům.

Dle~článku 16 má právo na~opravu nepřesných a na~doplnění neúplných osobních údajů.

\subsection*{Výmaz a~omezení}

Dle článku 17 má právo na~výmaz (\enquote{právo být zapomenut}), pokud jeho údaje již nejsou potřebné pro~deklarované účely, subjekt odvolává souhlas, vznese námitku proti~zpracování nebo jsou jeho údaje zpracovávány protiprávně. Toto právo se neuplatní, pokud je zpracování nezbytené pro~výkon práva na~svobodu projevu a~informace, pro~splnění právní povinnosti nebo pro~splnění úkolu provedeného ve~veřejném zájmu nebo při~výkonu veřejné moci nebo pro~určení, výkon nebo obhajobu právních nároků.

Dle článku 18 má právo na~omezení zpracování, pokud subjekt popírá přesnost osobních údajů (to na~dobu potřebnou k~ověření přesnosti správcem), zpracování je~protiprávní a~subjekt odmítá výmaz osobních údajů, správce údaje nepotřebuje, ale subjekt je požaduje pro výkon právních nároků.

\subsection*{Marketing, profilování a~automatické zpracování}

Dle článku 21 má právo vznést námitku proti~zpracování včetně profilování. Subjekt má vždy právo vznést námitku proti~zpracování pro~účely marketingu.

Subjekt má právo nebýt předmětem rozhodnutí založeného výhradně na~automatizovaném zpracování. Toto právo se nepoužije pokud to není nezbytné k~uzavření nebo plnění smlouvy, pokud je to povoleno právem Unie/členského státu nebo pokud je založeno na~výslovném souhlasu subjektu.

\clearpage
\section{Informace veřejného sektoru -- pojem, česká a~evropská právní úprava}

\clearpage
\section{Otevřená data -- pojem a~právní úprava}

Jde o~úplná, snadno dostupná, strojově čitelná data používající standardy s~volně dostupnou specifikací, která jsou zpřístupněna za~jasně daných podmínek užití s~minimem omezení, dostupná uživatelům při~vynaložení minima úsilí.

Zákon č.~106/1999 Sb., o~svobodném přístupu k~informacím, §3 (11): \emph{\enquote{Otevřenými daty se rozumí informace zveřejňované v~otevřeném a~strojově čitelném formátu, jejichž způsob ani účel následného využití není omezen a~jsou evidovány v~národním katalogu otevřených dat.}}

\begin{itemize}[noitemsep]
\item data jsou online
\item data jsou online ve~strukturované podobě
\item data jsou online ve~strukturované podobě v~otevřeném formátu
\item data jsou online ve~strukturované podobě v~otevřeném formátu a~mají vlastní IRI
\item data jsou online ve~strukturované podobě v~otevřeném formátu, mají vlastní IRI a~jsou přímo propojená s~dalšími datovými zdroji (linkování)
\end{itemize}

Povinná data jsou například jízdní řády, metadata registru smluv nebo příjemci dotací.

Mezi právními problémy jsou např. špatně nastavené smlouvy s~provozovateli dat (vendor lock-in -- \href{http://ictjudikatura.law.muni.cz/wiki/6_As_38/2015_-_51_-_\%C5\%BD\%C3\%A1dost_o_specifick\%C3\%BD_form\%C3\%A1t_informac\%C3\%AD_(CHAPS)}{spor CHAPS v.~Seznam o~jízdní řády}), smlouvy zastaralé nebo neexistující, práva duševního vlastnictví (autorské právo, právo pořizovatele databáze -- u~státních dat neexistují a~nic licencovat není třeba), anonymizace (dle GDPR).

Např.~\href{https://mapaexekuci.cz}{mapa exekucí}, \href{https://prazdnedomy.cz}{prázdné domy}, \href{https://hlidacstatu.cz}{Hlídač státu}, \href{https://data.brno.cz}{data.Brno}.
