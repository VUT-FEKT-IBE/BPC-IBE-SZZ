\section{Definujte a vysvětlete základní pojmy (aktiva, hrozba, ochrana, bezpečnost, zranitelnost, riziko, incident a~dopad).}

\newpage
\section{Bezpečná konfigurace přepínače a směrovače (základní postupy konfigurace, bezpečnostní funkce, útoky, Port Security, port Fast, hardening).}

\newpage
\section{Problematika logování, hlavní cíle a rozdělení (definice logu, základní kategorie, formát, obsah logu, struktura záznamu, ochrana logů).}

\newpage
\section{Definice operací nutných k aplikaci automatické analýzy logů (blokové schéma včetně popisu funkce jednotlivých bloků). Jakým způsobem je realizován blok korelace při detekci známých a neznámých událostí.}

\newpage
\section{Detekce nepříznivých událostí na základě signatur a~anomálií, systémy IDS/IPS (vzájemný vztah, efektivita a ladění, umístění, základní architektura, zástupci, referenční model).}

\newpage
\section{Dělení penetračních testů (dle znalosti, způsobu realizace a cíle), metodologie testování (pět kroků testování). Penetrační testování webových aplikací (OWASP, průzkum prostředí, závěreční report).}

\newpage
\section{(D)DoS útoky (princip, rozdělení, popis základních útoků, SYN Flood, HTTP Flood, DNS reflection, Ping of Death, Slowloris). Zátěžové testování (typy testů, nejznámější nástroje).}

\newpage
\section{Netechnické typy útoků, sociální inženýrství, phising (používané techniky), útoky MitM (ARP spoofing, DNS spoofing, SSL strip, SSL sniff).}

\newpage
\section{Protokoly IPsec a TLS (princip, umístění TCP/IP, průběh komunikace, autentizace, utajení a integrita dat).}

\newpage
\section{Zabezpečení 802.11 (WPA2, používaná kryptografická primitiva, klíčové hospodářství, popis 4Way Handshake, testování bezpečnosti).}

\newpage