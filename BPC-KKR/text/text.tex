\section{Vzájemný vztah pojmů kyberkriminalita, kybernetická bezpečnost a~kybernetická obrana.}

\newpage
\section{Prameny práva (národní, evropské i mezinárodní) obsahují hmotněprávní a procesněprávní úpravy kyberkriminality.}

\newpage
\section{Úmluva o kyberkriminalitě a směrnice o útocích na informační systémy (obsah úpravy a vztah k české právní úpravě)}

\newpage
\section{Postupy a kriteria při kvalifikaci trestné činnosti (vč. problematiky kvalifikované a privilegované skutkové podstaty)}

\newpage
\section[Kategorizace kyberkriminality (včetně příkladů trestné činnosti v jednotlivých kategoriích)]{Kategorizace kyberkriminality (včetně příkladů trest-né činnosti v jednotlivých kategoriích)}

\newpage
\section[Kyberkriminalita v užším smyslu slova (příklady trestné činnosti a~kvalifikace dle zvláštní části TZ)]{Kyberkriminalita v užším smyslu slova (příklady trest-né činnosti a kvalifikace dle zvláštní části TZ)}

\newpage
\section{Elektronické důkazy a jejich specifika v trestním řízení.}

\newpage
\section{Procesní nástroje pro zajišťování elektronických důkazů.}

\newpage
\section{Specializované útvary OČTŘ ve vztahu ke kyberkriminalitě.}

\newpage
\section{Mezinárodní spolupráce v oblasti kyberkriminality.}

\newpage