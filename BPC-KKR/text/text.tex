\section{Vzájemný vztah pojmů kyberkriminalita, kybernetická bezpečnost a~kybernetická obrana.}
Liší se v tom KDO CO provádí za JAKÝM účelem a za pomoci JAKÝCH prostředků.

\subsection{Kyberkriminalita}
Obecný pojem, který lze chápat v užším nebo širším smyslu - v užším smyslu do něj zahrnujeme \textbf{trestnou činnost}, která směřuje právě přímo proti důvěrnosti, dostupnosti, či integritě informačních systémů (neoprávněný přístup k počítačovému systému a nosiči informací, nebo opatření a přechovávání hesla a přístupového zařízení k počítačovému systému). 

V širším smyslu lze zahrnout další dvě kategorie trestné činnosti - \textbf{tradiční trestnou činnost}, která je páchána prostřednictvím informačních a komunikačních technologií - tedy například podvod, nebo výroba a jiné nakládání s dětskou pornografií, a \textbf{trestná činnost} při jejímž páchání je využito informačních a komunikačních technologií incidentálně, přičemž se při jejím vyšetřování dají čerpat elektronické důkazní prostředky.

Kyberkriminalitou se zabývají orgány činné v trestním řízení - tedy především policie, státní zastupitelství a soudy.  Primárními nástroji využívány orgány činnými v trestním řízení jsou procesní postupy definované trestním řádem - tedy především zajišťování elektronických důkazů pomocí odposlechů, zajišťování dat, forenzní analýzou, apod.

\subsection{Kybernetická bezpečnost}
 je aktivita, která spočívá v monitoringu počítačových infrastruktur, jejímž cílem je zjistit výskyt zranitelností těchto infrastruktur, případně kybernetických bezpečnostních incidentů. Kybernetická bezpečnost tedy mimo technických nástrojů zahrnuje rovněž nástroje právní a organizační. Jejím cílem je primárně zajistit CIA triádů (dostupnosť, integritu a dôvernosť informácií) spravovaných infrastruktur. Kybernetickou bezpečnost zajišťují na úrovni organizací takzvané CSIRT týmy, které sledují a zabezpečují svoji izolovanou infrastrukturu, na úrovni státu je to v případě ČR Národní a Vládní CERT tým, které koordinují postupy provozovatelů sítí a systémů. Na mezinárodní úrovni jsou to různé mezinárodní organizace a dobrovolné spolky, které především umožňují sdílení informací (například ENISA).
 
\subsection{Kybernetická obrana}
je obrana proti kybernetickým útokům, které mají charakter vojenského útoku, které objektivně ohrožují suverenitu státu. Kybernetická obrana státu by tak měla spočívat v budování schopností vojenských složek ubránit národní infrastrukturu proti takovým útokům a budovat za tím účelem dostatečně akceschopné jednotky. V ČR má na starosti kybernetickou obranu resort Ministerstva obrany a především Vojenské zpravodajství (jde o vojenskou rozvědku a kontrarozvědku).

\newpage
\section{Prameny práva (národní, evropské i mezinárodní) obsahují hmotněprávní a procesněprávní úpravy kyberkriminality.}

\subsection{Hmotněprávní úpravy}
Třestní zákoník č. 40/2009 Sb. ve znění pozdějších předpisů -- popis trestního práva hmotného - obsahuje
\begin{itemize}
    \item obecnou část (obecná definice pojmů, co je trestný
čin, jaké jsou možnosti trestu, urhný x souhrný trest, polechčující okolností, promlčení, kdo může
být pachatelem, zavinění (umysl x nedbalost))
    \item zvláštní část, kde je popis jednotlivých
skutkových podstat trestných činnů - jsou definovány skutkové podstaty týkající se přimo
kyberkriminality - neopravněný přístup k poč. systému §230, příprava hack nástroju §231 - nebo
jsou součástí skutkových podstat týkajících se běžné trestné činosti - porušení tajemstí §182,
porušení autorského práva §270, dětská pornografie §192, nebezpečné pronásledování §354,
podvod §209 - v těchto případech se často jedná o kvalifikovanou skutkovou podstatu nebot ICT
umožňují snažší páchání daného TČ s větším dopadem.
\end{itemize} 


\subsection{Procesněprávní úpravy}
Trestní řád - popisuje trestní právo procesní - definuje samotný proces a jeho fáze, zúčastněné
osoby, obviňěný, poškozený, obhájce a jejích postavení v třestním řízení - definuje zásady/postupy
trestního řízení/stíhání (procesní nástroje) - obecná součinnost §8, freezing §7b, odposlech §88,
zajištění provozních a lokalizačních údajů §88a odst. 1, sledování osob a věcí §158d - některé
nástroje jsou přímo spjaty s ICT (freezing, metadat), jiné jsou nejsou původně zamýšleny pro ICT
ale jsou tak nyní používané (sledování osob a věcí, pro získání dat od ISP).

Další prameny práva upravy kyberkriminality:
\begin{itemize}
    \item Zákon o elektronických komunikacích č. 127/2005 Sb
        \begin{itemize}
            \item Data retention -- uchování provozních a lokalizačních údajů po dobu 6 mesíců
        \end{itemize}
    \item Úmluva o kyberkriminalitě
    \item Směrnice o útocích na informační systémy
    \item Evropská úmluva o vzájemné pomoci ve věcech trestních č.
    550/1992 Sb. (mezinárodní trestní právo)
    \item Zákon o mezinárodní justiční spolupráci ve věcech trestních
    č. 104/2013 Sb. (mezinárodní trestní právo)
    \begin{itemize}
            \item definuje justiční spolupráci MLA - \textbf{Evropský vyšetřovací příkaz} (založen na uznávání příkazů - MR)
        \end{itemize}
    \item CLOUD act
\end{itemize}


\newpage
\section{Úmluva o kyberkriminalitě a směrnice o útocích na informační systémy (obsah úpravy a vztah k české právní úpravě)}

\textbf{Úmluva o kyberkriminalitě} je nejúspěšnější instrument práva mezinárodního veřejného v oblasti kyberkriminality - vydána roku 2001 - ČR podepsala roku 2005 - ratifikace roku 2013 - ratifikováno 65 státy - dálší
státy provedli třeba teprve část a zbytek je pro ně problematický implementovat nebo jim to trvá
déle - vpodstatě i u nás - až roku 2013 jsme dokončili implementaci, přijetím zákona o trestní
odpovědnosti právnických osob - velmi podobná je se \textbf{Smernicí o útocích na informační
systémy} - směrnice EU ale skutkové podstaty jsou formulované v podstatě stejně - tyto
dva úpravy jsou hlavní zdroje mezinárodního práva, které ovlivňují právní úpravu v ČR - jsou v jich
definované skupiny trestných činů - Zločiny proti důvěrnosti, interitě a dosažitelnosti systému -
Zločiny se vztahem k počítači a k přenášenému obsahu (zásahy do soukromí) - Zločiny se vztahem
k autorským právům

K Úmluvě o kyberkriminalitě byl podepsán první dodatkový protokol (kriminalizace činů
rasistické a xenofobní povahy spáchané prostřednictvím počítačového sytému) - který se týka
dětské pornografie - v Radě Evropy pracuje skupina odborníků na druhém dodatovkém
protokolu - který by měl řešit problém s omezenou možností spolupráce (nové procesní nástroje
pro exektivní předávání dat, videokonference, mechanismus předávání dat apod)

Vznikla kvůli problémům mezinárodního práva - snaží se tedy harmonizovat - podepsalo 44 států -
z nečlenských států pouze USA - udává, aby sankce byly účinné, přiměřené a odstrašující

Problém úmluvy je že je z roku 2001 a nereflektuje tedy aktuální stav - a dálší problém je že
nepokrývá vše, neboť nedošlo ke společnému konsenzu

Nastavuje procesní věci, které zjednodušují mezinárodní justiční spolupráci - sjednocuje procesní
nástroje - pro spolupráci je potřeba dvojítrestnost + musí mít nástroje pro vyhovění

Úmluva o kyberkriminalitě říká, že v právním řádu jednotlivých států (procesní nástroje úmluvy) musí být zajištěné:
\begin{itemize}
    \item urychlené uchování dat (freezing) - TŘ §7b
    \item  urychlené uchování a vydání provozních a lokalizačních údajů - má existovat vydávací příkaz -
    možnost prohledání a zajištění dat - v Zákoně o elektornických komikacích
    \item vydání věci - trestní zákoník
    \item  možnost odposlechu komunikace - TŘ §88
    \item musí existovat orgán, který dokáže ve dne v noci 24/7 tuto spolupráci realizovat
    \item  mechanizmus pro dobrovolné předávání informací - dále je v umluvě upraven způsob využití
    těchto procesních nástrojů - stále je to všechno postaveno na MLA - mezinárodní justiční
    spolupráci - úprava v trestním řádu
    \item možnost přímého přístupu k datům v zahraničí při souhlasu
\end{itemize}

\newpage
\section{Postupy a kriteria při kvalifikaci trestné činnosti (vč. problematiky kvalifikované a privilegované skutkové podstaty)}
Postupy a kriteria při kvalifikaci trestné činnosti:
\begin{itemize}
    \item zjištění zda jde o TČ nebo ne - OČTŘ - formální pojetí TČ - zjištění jestli skutkové okolnosti
    naplňují formální znaky skutkové TČ pospsaného v TZ ve zvláštní části -- \textbf{právní kvalifikace}
    \item okolnosti vyloučení protiprávnosti - některé TČ jsou trestné pouze za určitých okolností
    (místo, čas, způsob) - promlčení
    \item odpovědnost podezřelého - věk, příčetnost
    \item určení zavinění - úmysl a nedbalost
    \item zohlednění účinnosti Trestního práva - zásady jusrisdikce - teriotrialita, registrace,
    personalita, ochranná a univerzální
    \item zjištění zda jeden spáchaný skutek nenaplňuje znaky i dálších trestných činnů, subsumpce
    pod více TČ
    \item souběh TČ a jejich vyloučení - pachatel nesmí být potrestnán dvakrát za jeden čin
    \item určení míry trestu - kvalifikovaná skutková podstata, polechčující přitěžující okolnosti, úhrný
    a souhrný trest
\end{itemize}

\subsection{Právní kvalifikace}
\textbf{Právní posouzení určitého jednání} -- jde o rozhodnutí zda spáchaný skutek/čin byl nebo nebyl TČ pokud - byl tak daný skutek zařazen pod odpovídající TČ. 

Dále se posuzují trestněprávní odpovědnosti obviněného a okolnosti případného obvinění, nepříčetnosti obviněného, polehčující a přitěžující okolnosti a podobně - dále se určuje např zavinění (úmysl x nedbalost) - dále se řeší
souběh (a vyloučení souběhu, úhrný a souhrnný trest, zajištění aby pachatel nebyl potrestán
dvakrát za stejný čin). 

Kvalifikace začíná podmětem k TČ a končí soudním rozsudkem, který ukončí/provede konečnou kvalifikaci a udělí trest v případě odsouzení, může taky neodsoudit (shledat obžalovaného nevinným) - kvalifikace se v průběhu trestního stíhání může měnit - jednotlivá vyjádření jsou ve fázích zahájení (OČTŘ) TS - obžaloba (státní zástup) - rozsudek (soud)
a další - kvalifikace se tak může měnit (očtř, státní zástupce a soud na sebe nejsou závislé, každý
dělá svojí kvalifikaci)

\subsection{Skutek}
Činnost obviněného, která má za následek porušení nebo ohrožení společenských zájmů,
které chrání Trestní zákon - za jeden skutek se dají považovat různá stádia vývoje trestné činnosti
(příprava, pokud, dokonání TČ) - jeden skutek může mít znaky více trestných činů a nebo žádného
- né každý skutek je tedy třestným činem - skutek se TČ stáva pokud naplní znaky skutkové
podstaty některého trestného činu uvedeného v Trestním zákoně (TZ) - na základě skutku OČTŘ
prvně rozhodnou, zda se vůbec jedná o trestnou činnost - pokud ano, tak zahajují trestní
řízení/stíhání - pokud ne není dále co řešit.

\textbf{Právní kvalifikaci} chápeme jako subsumpci/zařazení skutku pod příslušné ustanovení TZ nebo
jiných předpisů z oboru trestního práva

\subsection{Trestný čin}
§13 TZ - říká, že TČ je to co TZ popisuje jako TČ - čin který naplňuje znaky
skutkové podstaty některého z TČ popsaných v TZ - k trestní odpovědnosti je potřeba úmyslného
zavinění nestanoví-li zákon výslovně, že stačí zavinění z nedbalosti - už máme pouze formální pojetí
TČ (problém pro etické hackery) - dříve formálně-materiální pojetí (spoločenská škodlivost)

Prvky trestného činu:
\begin{itemize}
    \item \textbf{objekt} (co je chráněno TZ | předmět je to na co útočí)
    \item \textbf{objektivní stránka} (popis co bylo spácháno a jaké to mělo následky + vztah mezi jednáním a
    následkem-příčinný vztah - kauzální nexus) - obligatorní znaky: jednání, následek, kauzální
    nexus(příčinný vztah) - fakultativní znaky: místo, čas a způsob jednání - na fakultativní
    znaky se musí ohlížet (některé TČ jsou jimi jen na nějakém místě v určitém čase nebo určitým
    způsobem jednání) můžou zapříčinit že daný skutek není TČ - Objektivní stránka nejzřetelněji
    odlišuje různé typy trestného jednání
    \item \textbf{subjekt} (kdo je pachatelem) - určen: věkem, příčetností, způsobilostí, postavením
    \item \textbf{subjektivní stránka} (zavinění, vztah subjektu k TČ | úmysl, nedbalost)
\end{itemize}

\subsubsection{Souběh TČ}
spáchání více TČ najednou (dřív než je jeden z nich odsouzen rozhodnutím soudu 1.
stupně).
\begin{itemize}
    \item \textbf{Jednočinný souběh} (též konkurence trestních zákonů) -- stav když je více TČ spácháno
    (konkurence více právních kvalifikací) nad jedním skutkem (jedním skutkem/činem)
    \item \textbf{Vícečinný souběh} -- souběh/konkurence více TČ nad více skutkami
    \begin{itemize}
        \item stejnorodý (stejné skutkové podstaty/stejné TČ)
        \item nestejnorodý (různé skutky podstaty/ různé TČ)
        \item kombinace:
        \begin{itemize}
            \item jednočinný stejnorodý
            \item jednočinný nestejnorodý
            \item vícečinný stejnorodý
            \item vícečinný nestejnorodý
        \end{itemize}
    \end{itemize}
\end{itemize}
Obecně se tresty TČ nesčítají (tomu tak je v USA, u nás ne) - posuzovat souběh je důležité, aby obviněný nebyl
trestán víckrat za jeden skutek pokud je jejich souběh vyloučen (zajištění aby pachatel nebyl potrestán dvakrát za stejný čin)

\subsubsection{Úhrný a souhrnný - úhrný trest}
Za jeden čin/skutek můžeme být potřestaný více skutkovejma podstatama - proniknu k poč. systému a tam získám udaje díky kterým padělám platební prostředek - trestá se podle závažnější skutkové podstaty a drží je horního limitu trestu a
dá se i trošku navýšit - zajištění aby pachatel nebyl potrestán dvakrát za stejný čin - souhrný trest
je když sem odsouzený za jeden TČ a pak jsem odsouzenej za dálší tak se dělá sourhný trest kde
se zase posuzuje ten závažnější TČ - obecně se tresty TČ nesčítají (tomu tak je v USA, u nás ne, u
nás se používá úhrný a sourhný trest)

\subsection{Postup pri kvalifikaci trestné činnosti}
\begin{itemize}
    \item Podnět -> přípravné fáze -> Rekognoskační fáze (jde opravdu o TČ?) -> zahájení vyšetřování ->
    vyšetřování el. místa činu -> vyšetřování fyzického místa činu -> zpracování důkazů -> dokazování
    před soudem
    \item \textbf{Zánik trestní odpovědnosti} - Účinná lítost - Promlčení (20 let, 15 let, 10 let, 5 let, 3 léta) - Vyloučení z promlčení
    \item police zajišťuje informace vedoucí ke kvalifikaci trestného činu a identifikaci pachatele
\end{itemize}
Zásady jurisdikce:
\begin{itemize}
    \item zásada teritoriality - TČ v ČR
    \item zásada registrace - TČ na plavidlech, letadlech registovaných pod ČR (pod českou vlajkou)
    \item zásada personality - aktivní - TČ spáchaný českým občanem - pasivní - TČ spáchaný na
    českém občanu
    \item zásada ochrany a univerzality - obecné zásady - okrajové prinicpy (moc nenastávají) - třeba
    když někdo páchá genocidu tak je to stíhatelné podle práva ČR vždycky, bez ohledu na to kde
    je to páchané
    \item Spáchání:
        \begin{itemize}
            \item Úmysl
            \item Nedbalost
        \end{itemize}
    \item \textbf{Formální pojetí TČ} - tč je to co je napsáno v TZ není potřeba škodlivosti/protiprávnosti
    \item Prípravné řízení -> předběžné projednání obžaloby -> hlavní líčení -> odvolací řízení -> výkon
    rozhodnutí
    \item Podezřelí -> obviněný (oznámení obvinění / zahájeno TŘ) -> obžalovaný (podání obžaloby státním
    zástupcem k soudu) -> odsouzený (soud rozhodl o vině a trestu)
    \item Trestní právo hmotné -> Skutkové podstaty -> trestné činy (trest)
\end{itemize}
Ke klasifikaci každý přistupuje trochu jinak (otázka \ref{kategorizace}).


\newpage
\section[Kategorizace kyberkriminality (včetně příkladů trestné činnosti v jednotlivých kategoriích)]{Kategorizace kyberkriminality (včetně příkladů trest-né činnosti v jednotlivých kategoriích)}
\label{kategorizace}
Kyberkriminalitu je možné kategorizovať z niekoľkých rôznych hľadísk. 

1. Obecne je kyberkriminalitu možné kategorizovať:
\begin{itemize}
    \item podľa trestného práva hmotného
    \item prostredníctvom klasifikácie bezpečnostných incidentov
\end{itemize}
2. Pomocou využitia \textbf{taxonomie} sa kategorizuje podľa prístupov:
\begin{itemize}
    \item Klasifikácia podľa CIA triády
    \item Klasifikácia podľa charakteru útočníka
    \item Klasifikácia podľa charakteru útoku
\end{itemize}
3 Kategorizácia podľa UNODC:
\begin{itemize}
    \item Útoky na CIA triádu - MITM, DoS...
    \item Trestné činy súvisiace s počítačom - spamming, vydieranie, zneužitie identity...
    \item Trestné činy súvisiace s obsahom - detská pornografia
\end{itemize}
4. Kategorizácia podľa využitia ICT prostriedkov:
\begin{itemize}
    \item \textbf{Cyber-dependent} - páchaná jedine prostredníctvom ICT. Kategóriu je ďalej možné rozdeliť na
    \begin{itemize}
        \item činy ohrozujúce CIA triádu
        \item využívajúce ICT
    \end{itemize}   
    \item \textbf{Cyber-enabled} - tradičné činy páchané prostredníctvom ICT
    \item \textbf{Cyber-supported} - Incidentálne využitie ICT (byli použity ICT a tak vznikají el. důkazy (telefonování))
\end{itemize}

\subsection{Taxonomie}
\begin{itemize}
    \item Stromová klasifikace pojmů - sdružuje skupiny incidentů a přiděluje jim skudkové podstaty odpovídající právní úpravy.
    \item ENISA/EUROPOL taxonomie - klasifikace podle charakteru útočníka a útoku
    \item Česká taxonomie dělení do skupin (Sběr informací, škodlivý kód, pokus o průnik, průnik, podvod) - v nich jsou definovány škodlivá jednání - klasifikace podle TZ v případě, že ji lze považovat za TČ - spam a scaning nejsou standardně trestné, ale za určitých podmínek už ano (spam s poplašnou zprávou nebo phishingem - taxonomie zároveň popisuje návody pro postup při výskytu určitého incidentu.
    \item OSN - UNODC - organizace OSN pro organizovaný zločin, kategorizace nic moc, prolínají se kategorie - Útok na CIA (hacking, MitM, DoS) - Trestné činy související s počítačem (Podvod, vydírání, zneužití identity, spamming, duševní vlastnictví, poškozování osob a skupin) - Trestné činy související s obsahem (Dětská pornografie)
\end{itemize}

\subsection{Cyber-dependent}
\begin{itemize}
    \item \textbf{Porušení tajemství dopravovaných zpráv (§182 TZ)} - útok na důvěrnost - i klasická pošta, email, neveřejný přenos počítačových dat. 3 roky vězení max, Zneužití pracovní pozice, např že jsem ISP tak je to kvalifikovaná TČ až 5 let, MitM
    \item \textbf{Porušení tajemství listin a jiných dokumentů uchovávaných v soukromí (§183 TZ)} - neoprávněné porušení tajemství - Kámoš nahlédne do mé soukromé složky, dokumentů
    \item \textbf{Neoprávněný přístup k počítačovému systému a nosiči informací (§230)} - 2 skutkové podstaty v jednom
    \begin{itemize}
        \item neoprávněné překonání bezpečnostního opatření -> získá přístup k počítačovému systému a tým spáchal trestný čin (CZ.NIC se skenoval síť proti defalutním nastavením kamer, ale tím páchali TČ. Když vím jaké heslo má kamarád a nebo mi ho klidně i řekne ale nedovolí mi se k němu připojit a já se připojím tak páchám TČ) = 2 roky
        \item po neoprávněném získání přístupu padělání nebo pozměnění dat = 3 roky -> úmysl omezit fungování (4 roky), pokud je to právnický subjekt, státní správy, podniku nebo státní moci (5let)
    \end{itemize}
    \item \textbf{Opatření a přechování přístupového zařízení a hesla k počítačovému systému a jiných takových dat (231 TZ)} Příprava hackingu je TČ - má to vlastní skutkovou podstatu, aby mě mohli potrestat když to budu připravovat někomu jinému - pokud je to pro výskum tak to trestné není.
    \item \textbf{Požkození záznamu v počítatčovém systému a na nosiči informací a zásah do vybavení počítače z nedbalosti (§232 TČ)} - pro profesionály - někdo odpovědný za bezpečnost a chová se drubě nedbale a způsobí to značnou škodu (>500 000) je to TČ - max 6 měsíců.
    \item \textbf{Neoprávněné opatření, padělání a pozměnění platebního prostředku (§234 TZ) - phishing do bank a platebních kare}t - trestná i příprava - získání něčích údajů a pak s nima dál nakládám (2 roky) - padělání (5 let) - sám použiju pro své obohacení (8 let).
    \item \textbf{Výroba a držení padětelského náčiní}
\end{itemize}

\subsection{Cyber-enabled}
\begin{itemize}
    \item \textbf{Výroba a jiné nakládání s dětskou pornografii (§192 TZ)} - bez distribuce el. sítí (2 roky) -
šíření ručně (3 roky) - šíření el. sítí (6 let)
\item \textbf{Porušení autorského práva (§270 TZ)} - prostřednictví počítačové techniky je to horší
\item \textbf{Nebezpečné pronásledování (§354 TZ)}
\item \textbf{Podvod (§209 TZ)} - phishing, sociální inženýrství, často spojené s něčím jiné
propagace drog, hanobení národa apod. - kvalifikovaná podstata jiných TČ
\end{itemize}




\newpage
\section[Kyberkriminalita v užším smyslu slova (příklady trestné činnosti a~kvalifikace dle zvláštní části TZ)]{Kyberkriminalita v užším smyslu slova (příklady trest-né činnosti a kvalifikace dle zvláštní části TZ)}
\textbf{Kyberkriminalita} -- v užším smyslu do něj zahrnujeme \textbf{trestnou činnost}, která směřuje právě přímo proti důvěrnosti, dostupnosti, či integritě informačních systémů (neoprávněný přístup k počítačovému systému a nosiči informací, nebo opatření a přechovávání hesla a přístupového zařízení k počítačovému systému).

\subsection{Příklady trestné činnosti}
\begin{itemize}
    \item \textbf{Scanning} - pouze pokud je to prováděno na účelem následného zneužití (§230, §182) - Opatření a
přechovávání přístupového zařízení a hesla k počítačovému systému a jiných takových dat §231
    \item \textbf{Sniffing} - Poručení tajemství dopravovaných zpráv §182 - Příprava nástrojů §231 - Neoprávněné
opatření, padělání a pozměnění platbního prostředku §234
    \item \textbf{Phishing} - Příprava nástrojů §231, získání např přístupových údajů za účelem §230 nebo §182 -
Podvod §209 - Platební prostředky §234
    \item \textbf{Ransomware} - Podvod §209 - Neoprávněný přístup §230 - Porušení tajemství listin a jiných
dokumentů uchovávaných v soukromí §183
    \item \textbf{(D)DoS} - Neoprávněný přístup §230 - jsou tam na to speciální písmena - v DDOS ještě dálší §230
aby získal botnet - teoreticky i §231 příprava nástoje pro následující útok
    \item \textbf{Využivání zranitelnosti/exploitu} - §230 při dokonání - §231 při neúspěšném pokusu je stále
dokonaná příprava (musel si ten exploit připravit) takže kvalifikace podle §231
\end{itemize}


\newpage
\section{Elektronické důkazy a jejich specifika v trestním řízení.}
Základní specifika/problémy:
\begin{itemize}
    \item Neexistuje specifická úprava pro práci s el. důkazy -> používají se standardní nástroje TP procesního
    \item Nutnost přenesení důkazu do vnímatelného zachycení, aby se sním mohl seznámit soudce - el. důkazy se těžko převádí do vnímatelného zachycení - např dokazování zdrojovým kódem nebo metadaty se nedá jentak samo použít - soudce by nevěděl do čeho kouká - nutnost interpetace
    \item Nedostatek/absence obecně uznávaných postupů - ani judikatura (soudy moc ještě nerozhodovali o specifikách el. důkazů) - OČTŘ nemá jasně definovaný postup jak s el. důkazy nakládat - na různých úrovních nebo v ruzných krajích se to dělá jinak
    \item volatilita el. důkazů - rychle se menící prostředí - data mohou mizet - proto máme nástroj freezing
    \item často přeshraniční charakter
    \item nedostatek vzděláných lidí, znalci, zástupci, soudci - předpokládá se vyšší úroveň zločinců(chytřejší, organizovanější), než jen ty co se vloupávají do chat apod
\end{itemize}

\textbf{Základní zásady dokazování = presumpce neviny - zásada ústnosti} (ústně a prezenčně se
důkazy předávají soudci a osvšem okolo, problém s el. důkazy) - \textbf{zásada veřejnosti - zásada
bezprostřednosti} (soudce je seznámen s důkazy bezprostředně, problém s el. důkazy, né
všechno se dá přinést k soudu, potřeba vymyslet mechanizmus jak soudce co nejbezprostředněji
s důkazem seznámit) - \textbf{zásada materiální pravdy - zásada vyhledávací - zásada volného
hodnocení důkazů} (neexistuje dělení důležitosti důkazů, vždycky je to subjektivní hodnocení,
problém s el. dukazy, když soudci někdo pořádně nevysvětli jak byl důkaz vygenerován, co to
znamená a jak je to spolehlehlivé je to pro něj vělmi těžké volně hodnotit a tak může kvalitní el.
důkaz diskreditovat) - \textbf{zásada zdrženlivosti} (redukce škod, například při zajišťování důkazů při
domovní prohlídce)

\subsection{Důkaz}
Jakákoliv informace, kterou získáváme - vyvracuje nebo potvrzuje skutkovou
okolnost - může to být cokoliv co může přispět k objasnění věci - není nikde definované co to
může být.

Musí být získán \textbf{zákonným postupem} (získán důkazním prostředkem). \newline
Dokazuje se: 
\begin{itemize}
    \item zda se stal skutek, který považujeme za TČ - kdo ho spáchal (zda to byl obviněný)
    \item z jakých pohnutek (důvodu), aby se dalo rozhodnout jestli skutek nenaplňuje kvalifikovanou skutkovou podstatu
    \item následky TČ (co bylo napácháno za škody a že je mezi nimi kauzální nexus) - poměry obviněného - okolnosti vedoucí ke spáchání nebo okolnosti (spolupachatelné, nedbalosti apod)
\end{itemize}
Dělení důkazů:
\begin{itemize}
    \item Osvědčující X Vyviňující
    \item Původní X Odvozené - svěděk, který to viděl (původní) - svědek, který se o tom od někoho dozvěděl (zprostředkovaný svěděk/důkaz | Odvozený)
    \item Přímé X Nepřímé - příme vypovýdají o zkutečnosti, nepřímé je že se někdo z daného počítače připojoval k serveru, ale přímo se neví (musí se dávat do souvyslostí, víc nepřímích)
\end{itemize}
U každéno důkazu se posuzuje jeho důkazní moc (síla) a to v podobě volného hodnocení důkazů
Složky:
\begin{itemize}
    \item \textbf{Závažnost důkazu} - do jaké míry dokazuje přímo či nepřímo danou skutečnost
    \item \textbf{Pravdivost důkazu} - důvěryhodnost zdroje důkazu, jak moc se můžeme spolehnout na pramen důkazu a na spolehlivost získání důkazu - analýza dat pomocí AI (každý přisoudí subjektivně jinou pravdivost/věrohodnost)
    \item \textbf{Zákonnost důkazu} - zde už spíše objektivně:
    \begin{itemize}
        \item  zda byl důkaz zajištěn v souladu s příslušnými předpisy (zda nebyl získán protiprávně) -> jakýkoliv protiprávní prvek při získání může věst k jeho neplatnosti (např pokud by policie daný pramen důkazů ukradla (telefon))
        \item úkon musí být proveden oprávněným subjektem - zda existuje vztah důkazu a skutku - zda byl zajištěn přípustným způsobem (nepřípustný důkaz = nesmí se k němu přihlížet, ani OČTŘ ani pak soudce, jako kdyby neexistoval)
        \item Podstatné X Nepodstatné vady -- některé vady jsou opravitelné (napravitelné X nenapravitelné vady) \newline
        Odposlech na základě soudu nebo na základě souhlasu účstníka komunikace, ale souhlas musí být protokolovaný, pokud není protokol, tak je důkaz neplatný dokud ho nedoplním - \textbf{Doktrína otráveného stromu} (hlavně v USA), všechny navazující důkazy jsou neplatné (odposlechem zjsitím heslo a pomocí toho se dostanu k něčemu = neplatné všechno), u nás to není tak striktní, ale furt to není vyřešené jak moc to platí nebo ne
    \end{itemize} 
\end{itemize}

\textbf{Vyhledávání důkazů} (často předchozí důkazy, provozní a lokalizační údaje mi řeknou kde zdroj) ->
Procesní zajštění důkazního materiálu (nejproblematičtější z hlediska generování zákonných důkazů) -> prověrka (získání důkazu z důkazního materiálu/pramene důkazu) -> volné hodnocení (jak OČTŘ i soudce, zda je zákonný, pravdivý) \newline
Procesní zajištění důkazů - musí nejprve zohlednit 2 věci - charakter důkazu(dat X hardware apod) - odkud budou zajišťována - na základě těchto 2 kritérii je zvolen procesního prostředku (nástroje) - musí se zohledňovat základní zásady dokazování (přiměřenost postupu - pokud můžu získat důkaz procesně složitějším způsobem, ale který zasahuje míň do práv, musím zvolit ten který zasahuje méně - nedělat odposlech/domovní prohlídku pokud to jde jinak)

\subsubsection{Zajištění el. důkazů}
\begin{itemize}
    \item \textbf{Zajištění zařízení či datového nosiče (získám hardware a tím se dostanu k el. důkazům)} - Vydání a odnětí věci - osobní prohlídka - domovní prohlídka a prohlídka jiných prostor -ohledání věci
    \item \textbf{Přímý přístup k datům prostřednictvím (třeba počítačové sítě, nebo telefon přijonená někam)} 
    \begin{itemize}
        \item Orgány můžou přístoupit k lokálním datům přímo bez souhlasu, pokus jsou data součástí odňatej věci
        \item Orgány můžou přístoupit k vzdáleným datům přímo bez souhlasu, pokud znají přístupové údaje (našli je napsané nebo od manželky), potřebný souhlas soudu
        \item K datům na připojených službách:
        \begin{itemize}
            \item Podle postupu §158 odst. 3  (Sledování osob a věcí) -- aktuální data (nutné pořídit protokol k dokazování)
            \item Podle postupu §88 (Odposlech) -- budoucí data
        \end{itemize}
        V obou případech je taktéž potřebný souhlas soudu
    \end{itemize}
    \item \textbf{Přístup prostřednictvím držitele/správce dat} - požádání toho kdo je má v držení nebo je spravuje aby mi ty data zpřístupnil nebo dal - obecně mě to mají povinnost vydat - specificky pak potřebuju odposlech nebo sledování. \newline
    \textbf{Data od ISP}
    \begin{itemize}
        \item provozovatele podle Zákona o elektronických komunikacích (provozovatele sítí,operátoři)  - provozovatele obvykle žádná data neuchovávají, ale podle data retention musí uchovávat metadata (provozní a lokalizační údaje) a poskytnout odposlech
        \item provozovatele podle Zákona o některých službách iformační společnosti (poskytovatele služeb na internetu (FB, google, seznam)) 
    \end{itemize}
    \item Podle charakteru dat se používají následující procesní postředky:
        \begin{itemize}
            \item Dožádání (obecná součinnost) -- nelze stáhnout na utajované -osobní údaje, údaje uchovávané v soukromí, metadata, obsahové data komunikací protože mají specifickou úpravu
            \item  sledování osob a věcí pro data uchovávaná v soukromí (data hosting)
            \item data retention pro zajištění metadat -- útvar zvláštních činností na základě rozhodnutí/příkazu soudu, operátor/poskytovatel služby musí tyto provozní a lokalizační data uchovávat půl roku
            \item odposlech pro přenášení/komunikační data -- realizován útvarem zvláštních činností -soud na základě zádosti státního zástupce vydá rozhodnutí o odposlechu, jakákoliv služba (ne jen telefon, email ale i sledování kom. PC nebo sledování účtu a sledování přibývajících ubývajících dat), max 4 mesíce, pokud souhlas uživatele, tak není potřeba souhlas soudu, vypracovává se protokol, z hlediska procesního práva policii nic neprání v tom se do komunikace vbourat např do zašifrované, odposlech možný i dobudoucna (i když neví zda TČ probíhá) musí to však být stále dobře odůvodněné
            \item  freezing (zamražení dat) -- aby nebyla zneužívána volativita el. důkazů, utajované pro uživatele, max 90 dní
        \end{itemize}
\end{itemize}

\subsubsection{Hodnocení a provedení důkazu}
Jedná se o poslední klíčovou fázy v procesu dokazování - důkaz musí být \textbf{smyslově vnímatelný}
pro soudce (velký problém při el. důkazech) a musí jednoznačene/dostatečně/srozumitelně
vypovídat o skutečnsoti kterou má dokazovat. \newline
Při dokazování datama se proto využívají interpretační prostředky (data uložená na disku mi jsou k ničemu pokud nemám interpretační prostředek, který mi je zobrazí do vnímatelné podoby) - např když dokazuji kódem, tak jako
interpretační prostředek použiju znalce, který vysvětlí co v tom kódu je a co dělá, nebo kód
zkompiluju a dokazuju vzniklím počítačovým programem nebo webovou stránkou apod - pokud
dokazuju datama v určitém datovém formátu, potřebuju sofwarové/harwarové vybavení, které ty
data interpretuje do srozumitelné, člověkem vnímatelné podoby - to je zase problém protože když
budu mít nějaký proprietární firmware čipu z auta třeba, tak pro interpretaci budu potřebovat
specifický forenzní nástroj nebo to potřebuju nějak nainstalovat do toho auta a to přinést k soudu
:) - pak se musí dokazovat i že ten interpretační prostředek interpretuje správně a nebo to to co
vzniklo kompilací kodu dělá to co autor původně zamíšlel - třeba webové stránky se mohou na
každém PC chovat jinak - interpret to tak pak může zkreslovat, ale soudy se k tomu zatím moc
nevyjádřili.

Často se řeší \textbf{znaleckými posudky} - odborníci v oboru s technickým vybavením - snaží se
vysvětlit/odpovědět na technické otázky u soudu - dostane počítač a otázka je zda na tomto počítači
prokazatelné že uživatel páchal TČ - znalec odpoví ano nebo ne a odůvodní odpověd s důkazy -
jmenuje to předseda soudu - není zádná znalecká komora - není žádná certifikace znalostí/
odborné způsobilosti - znalci se vyjádřují k právním otázkám - často se stává, že soudy ani neumí
formulovat zadání znaleckého posudku -v tom případě se spolupracuje se znalci na formulaci
otázky, na kterou pak odpovídají (potenciálně problematické).

\textbf{provedení důkazu }- bezprostřednost, ústnost, veřejnost - často se předčítají znalecké posudky a
jiné důkazy - často se znalec musí učastnit soudu, bývá vyslýchaná a doptávají se ho na další
detaily - obecně je problém že se to tak musí dělat a v soudních síně nedisponují dostatečnou
technikou na interpretaci, nebo nelze tam přijet s autem (příklad s čipem v autě) - takže se to
obvykle řeší znaleckými posudky

\subsection{Důkazní prostředek}
Prostředky pomocí kterých můžeme zjistit stav věci - to co může u
soudu reprezetnovat důkaz ve smyslově vnímatelné podobě - výsledek procesního postup
OČTŘ při zajištování důkazů - např výslech svědka, znalecký posudek - obecně to může však
být cokoliv
\subsection{Pramen důkazu}
Věc, ze které je důkaz čerpaný - např datový nosič, PC, dokument, metadata dokumentu, software, softwarové logy, hardware

\newpage
\section{Procesní nástroje pro zajišťování elektronických důkazů.}

\subsection{Obecná součinnost §8}
Státní orgány, fyzické a právní osoby a dálší relevantní subjekty mají povinnost vyhovovat na
dožádání - OČTŘ chce informaci a tak provede dožádání a daný subjekt by na tuto žádost měl
vyhovět - pokud existuje specifická právní úprava musí se použít ta - plus soudy pak řeší
proporcionalitu se zásahem do práv suběktu - zjišťování nejzákladnějších informací (obvykle v
rekognoskační fázy) - pro utajované informace je potřeba příkaz(souhlas) soudce (§8/5 "paragraf 8
odstavec 5)
\subsection{Freezing §7b}
Vyžaduje to po nás Úmluva o kyberkriminalitě - 2 procesní nástroje
\begin{itemize}
    \item \textit{freezing} -- hrozí ztráta zníčení nebo pozměnění dat důležitých pro třestní řízení, lze nařídít osobě která je drží, aby je uchovala v nezměněné podobě pro potřeby dálšího vydání vyšetřovatelům (při phishingu zamrznutí nasbírané databáze)
    \item \textit{blocking} -- příkaz na provozovatele služby, aby zablokoval přístup užívatele k daný datům (max 90 dnů), poskytovately služby (ten kdo freezing provádí) musí být vysvětleno co má být zmraženo, proč a na jak dlouhou dobu
\end{itemize}

\subsection{Odposlech a záznam telekomunikačního provozu §88}
při vedení TŘ pro zločin s odnětím svobody min. 8 let nebo pro vyjmenované TČ nebo pro
umyslné TČ k jehož stíhání nás zavazuje mezinárodní smlouva
\begin{itemize}
    \item může být vydán příkaz (úkon) pro zajištění obsahu telekomunikačního provozu (tekoucí data, e-maily telefonáty) - zajištěno \textbf{Útvarem zvláštních činností} 
    \begin{itemize}
        \item mívají nainstalované zařízení, které umožnuje tento odposlech
        \item odposlech realizují ve spolupráci s operátorama a ti za to dostávají peníze
    \end{itemize}
    \item pokud nelze sledovaného účelu dosáhnout jinak nebo pokud by to bylo moc složité (preferují se
jiné úkony) - potřeba příkaz soudu - nebo i bez soudu se souhlasem uživatele - odposlech možný
vydat i dobudoucna (i když OČTŘ neví zda TČ probíhá) musí to však být stále dobře odůvodněné
\end{itemize}

\subsection{Zajištění provozních a lokalizačních údajů §88a odst. 1}
metadata k tekoucí komunikaci - na základe příkazu soudu - pokud úmyslný TČ min 3 roky nebo
vyjmenované nebo TČ vyhlášený mezinározní smlouvou - pokud učelu nelze dosáhnout jinak -
nutno zajistit aby tyto údaje byly uloženy poskytovatelem - na to máme úpravu Zákona o
elektronických komunikacích

\subsection{Data retention §97 - Zákon o elektronické komunikaci (ZoEK)}
ZoEK říká co jsou provozní a lokalizační údaje:
\begin{itemize}
    \item \textbf{metadata o komunikaci} - údaje by neměli nic říkat o přenášených datech (poskytovatel by třeba měl odfiltrovat data obsažená v URL třeba z formulářů)
    \item  poskytovatel je na základě data retention povinen uchovávat provozní a lokalizační údaje po dobu 6 měsíců od doby uskutečnení daného komunikačního provozu (za to poskytovatel dostává peníze), nesmí být poskytovatelem zneužita. Pokud nejsou po 6 měsících vyšetřovately požadovana musí je tekomunikační operátor zkartovat
\end{itemize}

\subsection{Sledování osob a věcí 158d trestního řádu}
Pátrací prostředek - primárně určen pro zajištění operativních informací -> zjistit co se
stalo, ale jé na zjištění důkazů pro soud (tak tomu bylo původně)

Pokud chci zjištěné informace použít jako důkaz u soudu musím o tom vypracovat protokol - dnes se to používá i pro
zajištění el. dat jako důkazů. Postup:
\begin{enumerate}
    \item vyšetřovatel potřebuje data z uložiště -> zahájím sledování osob a věcí
    \item jako součinnost si vyžádám spolupráci od poskytovatele služby, který data uchovává -> a v rámci té součinnosti mi poskytně i ty data, která chci
    \item když udělám protokol o tom jak jsme získal danou spolupráci, kdo mě poskytnul součinnost, jak jsem postupoval a jaká data jsem získal -> mužu daná data použít jako el. důkaz u soudu
\end{enumerate}
O vydání příkazu o sledování osob a věcí rozhoduje státní zástupce prostřednictvím povolení - pokud jsou to ale soukromá nebo utajovaná data je potřeba předchozí povolení soudce - dá se udělat neodkladný úkol (získám telefon a neodkladně se kouknu na data uložená na vzdálené službě) pokud pětně bude soud souhlasit.

\subsection{Domovní prohlídka a prohlídka jiných prostor}
Postup:
\begin{itemize}
    \item Soudce vydá příkaz k realizaci domovní prohlídky
    \item  policejní orgán tam provádí ohledání věcí relevantní k trestnímu stíhání - policejní orgán musí soudci dostatečně vysvětli proč se domnívá že v příslušných prostorách jsou veci důležité pro TŘ
    \item soudní zástupce sepíše podání na soud ve kterém žádá o vydání příkazu k domovní prohlídce ve které uvede co je cílem a proč a proč si myslím že to tam bude a oduvodním že to nejde jiným nástrojem
    \item  soud to posoudí a v rámci rozhodnutí pak vypíše informace na základě kterých se rozhodoval a tím oduvodní vydání daného příkazu - nedá se odvolat, ale stát ručí za škody způsobené nesprávným vydáním příkazu o domovní prohlídce
\end{itemize}
Procesní podmínky za kterých může být domovní prohlídka realizovaná:
\begin{itemize}
    \item přiměřenost -  odborná péče buď vyšetřovatel nebo znalec (když to dokážou ohledat na místě nemusí se to odvážet)
    \item potřeba přítomnosti majítele prostor nebo musí být aspoň informován o tom co se tam děje
    \item nezúčastněná osoba (někdo externí, soused) kdo zkontroluje že nedochází k porušení zákona při prohlídce
    \item vypracovává se protokol (videozáznam, foto) - všichni kdo se zučastní ho podepisují i nezúčastněná osoba
\end{itemize}
Aby přistihly zločince se zapnutím PC a přihlášením do vzdálené služby - dá se přizvat znalec pro specifickou
činnost se specifickým vybavením - musí se dodržovat specifické postupy aby nebyly důkazy znehodnocovány - zapečetění, zabalení do pytle za přítomnosti nezúčastněné osoby.

\subsection{Výdání a odnětí věci}
Nástroj pro zajištění věci od člověka - každý má Ediční povinnost (na požádání musí vydat drženou věc) - OČTŘ ho vyzvou k předložení věci - pokud odmítně věc vydat může mu být odejmuta (pořádkové opatřené) stačí rozhodnutí státního zástupce - při odejmutí věci by měla být přítomna nezúčastněná osoba - pořizuje se protokol - osobě se dá potvrzení o odejmutí - pokud jsou na zařízení data o kterých je povinná mlčenlivost (utajované informace, advokátní tajemství) je specifický postup - pouze věci né data - ovšem pokud je vydán např telefon tak naněm může být provedena forenzní analýza a tedy je možno se dostat k datům uloženým na zařízení

\subsection{Osobní prohlídka}
domnívám (jako OČTŘ) se že daná osoba má u sebe věc důležitou pro TŘ, ale nevím to jistě -> zahájím osobní prohlídku - na základě rozhodnutí soudu nebo státního zástupce - pokud je to neopakovatelný úkon můžu osobní prohlídku provéset i bez příkazu (zatknu osobu co utíká z místa činu a mám podezření že u sebe má zbraň se kterou páchal TČ) musím pak ale souhlas zpětně získat - je to zhojitelná vada neučinného důkazu - úkonu by měl předcházet předchozí výslech (jako u domovní prohlídky) což bych osobu měl požádat zda věč/předmět u sebe má a zda mi ho nevydá.

\subsection{Ohledání věci}
pozorování a sbírání informací za účelem objasnění věci - protokol co bylo vypozorováno a jak - typicky při domovních a osobních prohlídkách - na místě najdu spuštěný počítač a na místě chci provést jeho ohledání - v takovém případě s kamerou nebo foťákem provádím záznam toho co jsem objevil.

Nemůžu provést obecné ohledání věci - např. když ohledávám na místě zaplé PC a bude tam probíhající komunikace a já bych ji chtěl sledovat (odposlech má větší zásah do práv), tak to nemůžu udělat jen na základě ohledání věci (generovalo by to neúčinný důkaz) - avšak policie si proto může předem připravit příkazy- například existují příkazy k domovní prohlídce, odposlechu a sledování osob a věcí a tím pádem může získávat všechno na místě.

\newpage
\section{Specializované útvary OČTŘ ve vztahu ke kyberkriminalitě.}
\subsection{Policie ČR}
\begin{itemize}
    \item\textbf{ Policejní presidium} -> Skupina informační kriminality - dřív více - hlavní organizační prvek
Policie ČR - koordinace a metodika - cílení na konzistenci pri vyšetřování.
 \item \textbf{Odbor kriminalistické techniky a expertíz (OKTE PČR)} - znalecký ústav - akreditace na
digital forensics (získávání el. důzaků z dat) - znalecká činost - zpracování znaleckého
posudku - málo lidí a tak tyto služby nabízí soukromnící kteří jim tyto služby prodávají
(komerční činost).
 \item \textbf{Kriminalistický ústav Praha} -> Oddělení počítačové expertízy - také znalecká činost ale jako
externí služba
 \item \textbf{Útvar zvláštních činností Policie ČR (UZČ SKPV PČR)} - centrální útvar policie v Praze -
expozitůry v jednotlivých krajích - poslituje vyšetřovatelům služby - odposlech a zajišťování
dat od poskytovatelů a poskytovatelů služeb (provozní a lokalizační údaje) - má k tomu
kontaktní sítě a vybavení - vyšetřovatel požádá soud o příslušný příkaz, ten dá utvaru
zvláštních čiností, který daný úkon provede (zajístí odposlech, zajístí data) a pak je v
protokolu předává - často postupují striktně podle zákonného postupu a tím se proslužuje
doba vyšetřování
 \item \textbf{Národní centrála proti organizovanému zločinu} - spojením útvaru pro odhalování
organizovaného zločinu a útvaru pro odhalování finanční kriminality - vyšetřování
nejzávažnějších druhů kriminality - včetně rozsáhlých kybernetických útoků (rozsáhlé
ransomwéry, útoky na kritickou infrastrukturu) - expozitury v jednotlivých krajích - 2 role -
vyšetřování - vytváření metodiky pro ostatní vyšetřovatelé/útvary policie (jaké techniky, jaké
nástroje používat při vyšetřování)
 \item \textbf{Analogická pracoviště na krajských ředitelstvích PČR (oddělení kybernetické kriminality,
odbor poč./inform. kriminality)} - různé jméno v různých krajích - vyšetřování na lokální úrovni
- neúspěšnější jihomoravský kraj, snaží se předávat znalosti do ostatních krajů
 \item \textbf{Europol -> EC3 jednotka} - Evropské centrum pro boj proti kyberkriminalitě - metinárodní -
nevyšetřují - podávají metodiku - sestavují mezinárodní týmy
 \item \textbf{Evropská agentura pro bezpečností sítí a informací} - ENISA
 \item \textbf{specializovaná pracoviště na Interpolu} - mezinárodní - nevyšetřují - podávají metodiku -
sestavují mezinárodní týmy
\end{itemize}

\subsection{Státní zastupitelství}
neexistují zvláštní specifické instituce/zástupci - pouze neformální skupiny na urovní nejvyššího
státního zastupitelství nebo na úrovni krajského státního zastupitelství - předávání metodiky a
snaha o zajištění koordinace a vzdělávání
obecne vělký problém nevzdělanosti - neexistuje jak je donutit se vzdělávat - dělají to dobrovolně
\subsubsection{Nejvyšší státní zastupitelství (NSZ)}
významná role, že může vydávat metodická opatření - dokumenty doporužující metodiky jak by
se mělo co dělat

metodický pokyn 1.2015 - popisuje, který procesní nástroje na které el. důkazy 

\subsubsection{Justiční akademie}
příspěvková organizace státu - poskytuje nadstandardní vzdělávání soudům a státním zastupitelstvím

Školení NÚKIB - semináře NSZ - akademická sféra - Ústav pro kriminologii a sociální prevenci, úzká
vazba na ministrestvo spravedlnosti a na nejvyšší patra policie (zatím jen statistické studie)
\subsection{Soudy}
neexistují ani neformální skupiny

\newpage
\section{Mezinárodní spolupráce v oblasti kyberkriminality.}
Problémy:
\begin{itemize}
    \item neharmonizovaná legislativa - využívání bezpečných přístavů - negativní kolize - kyberprostor
nemá hranice ale právo jo
\item složitá a pomalá harmonizace - chceme aby se harmonizace dotýkala co nejvíce států ale o to
těžší je se dohodnout na slopečné věci
\item Neochota některých států spolupracovat - fragmentace internetu - něbo některé státy nejsou
dostatečně vyspělé a nemají lidi na to aby spolupráci poskytli
\item pomalá spolupráce - obecné poskytnutí spolupráce (MLA) trvá příliž dlouho k volatilite el.
důkazů a státy nemusí spolupráci vůbec přijímat - pachatel se může rychle přemisťovat (i se
serverem)
\item neexistence nástroje pro el. předávání důkazů
\end{itemize}

Právo je teritoriální - každý stát vykonává svou suverenitu na svém území - snaha o mezinárodní harmonizaci/standardizaci - ale státel právo platí lokálně (omezeně teritoriálně pravomoci OČTŘ).
Teritoriální charakter práva nesedí s charakterem informačních sítí - geolokace a georestrikce funguje spíše v soukromoprávních vztazích (Neftlix nabízí ruzné filmy pro ruzné státy).

Většina kyberkriminality má přeshraniční prvek - nekdě je přeshraniční prvek natolik velký že se
jim OČTŘ musí zabívat - pachatel a poškozený v jiných státech - pachatel a poškozený ve stejných
státech ale vetšina dat souvsejících s TČ (důkazu) je v zahraničí - často to má dopad na výsledek TŘ
(získání důkazu trvá nebo se k nim nelze vůbec dostat apod) - při přeshraničním charakteru klesá
účinnost vyšetřování.

Často hrají velkou roli i zahraniční poskytovatelé služeb (ne jen stát), to má zase jiný charakter ne
jednání státu se státem.

Hlavní právní problémy - podle kterého práva? - co je a není trestný čin? - kdo a jak má získávat
důkazy a provádět úkony a jakým způsobem? (Policie ČR nemuže provádět úkony v zahraničí) - jak
spolupracovat?

\subsection{Podle kterého práva}
\begin{itemize}
    \item \textbf{suverenita} -- právo státu vykonávat státní moc nad určitým územím (teritorium, stát, lodě, vzduch)
    \item \textbf{jurisdikce} - právo státu definovat povinnosti a práva lidí - státy se obvykle staží rozšířit svou jurisdikci
    \item \textbf{rozhodné právo} - pravidla která říkají že na určitý případ se vztahuje určité právo (komu to spadá do jurisdikce) -> do určité míry mezinárodními úmluvami - vetšina států má jimi implementovanou jurisdikci ve svém právním řádu (problém je s výkladem, i když to mají implementované stejně tak dochází ke problémům)
\end{itemize}
V soukromém právu si můžeme zvolit podle kterého práva chceme postupovat a soud pak podle něj soudí - ve veřejnoprávních odvětvích jako je trestní právo to tak není - když určíme že v nějakém případě platí právní řád nějakého státu, začnou to řešit/rozhodovat orgány daného státu a to dělají podle jejich práva, né jiného (nemůžou si vybrat).

Vymezení jurisdikce ČR na základě principů: 
\begin{enumerate}
    \item \textbf{princip teritoriality} -- vztah k prostoru ČR -> např v jednom státě může být provozován malitious server, v druhém může být skupina lidí koordinující útok a ve třetím státě muže být dopad útoku (všechny tři státy si mohou nárokovat jurisdikci podle principu teritoriality)
    \item \textbf{pricip personality} -- aktivní (občan ČR spáchal) a pasivní (vůči občanu ČR byl ospácháno)
    \item \textbf{princip registrace} -- letadla a plavidla pod vlajkou ČR spadají pod jurisdikci ČR
    \item \textbf{pricip ochrany a univerzality} -- obecné principy, okrajové principy (moc nenastávají) -> třeba když někdo páchá genocidu tak je to stíhatelné podle práva ČR vždycky, bez ohledu na to kde je to páchané - požádání cizého státu který na má připad v jurisdikci aby byl případ odtíhán ČR tak se to bude řešit podle Českého práva
\end{enumerate}

Z těchto principů je vidět že jurisdikci nad určitým prípadem si může nárokovat více států najednou (kolize jurisdikcí) - kolize jurisdikcí se řeší z pravidla domluvou nebo to řeší nějaká mezinárodní organizace - určí si kdo to bude stíhat, nebo založí společný vyšetřovací team a stíhají to ve spolupráci v obou státech. Může vzniknout:
\begin{itemize}
    \item pozitivní kolize (více státu si nárokuje jurisdikci)
    \item negativní kolize (nikdo si nenárokuje jurisdikci)
\end{itemize}
Pachatel muže distibuovat svou činost tak aby to co zrovna páchá v konkrétním státu nebylo trestné - tím využije \textbf{bezpečného přístavu} daného státu - takhle může poschovávat jednotlivé činosti a vyhnout se tak stíhání - jediné řešení je mezinárodní spoluprací a to tak že se dohodnou že to budou upravovat stejně (to co je trestné u nás je trestné i u ostatních) = \textbf{harmonizace}

\subsubsection{Harmonizace v TP}
Společná definice toho co je a není trestný čin (definice hmotného práva trestního) tak i to jak se stíhá (definice procesního práca) - pro vydání důkazu od jiného státu je většinou potřebná dvojí trestnost (TČ v obou státech), jinak nemusí vyhovět vydání daných důkazů - pokud by se u nás stal TČ ale důkazy by byly v jiném státě, tak náš stát by popsal skutkový stav (co se stalo), jaké důkazy potřebuje a proč, pokud by to nebylo trestné i v druhém státě tak by nemohl použít jejich procesně správní prostředky a tak nemůže pomoct - proto se snaží co nejvíce harmonizovat (i procesní nástroje, aby vůbec daný důkaz mohl druhý stát zajistit)

Harmonizace je velmi složitý proces - vyždycky je tam stát, kterému se něco nelibí a tak to bojkotuje a tím se to buď vůbec nedá harmonizovat nebo to trvá strašně dlouho.

Harmonizační instrumenty (nástroje pro dosžení harmonizace):
\begin{itemize}
    \item Mezinárodní úmluvy (smlouvy) - drívě na rozdělení vod, vesmíru a vzdušného prostoru -
velmi komplikované dosáhnout společného řešení - Úmluva o kyberkriminalitě (2001 v
Budapešti), původně jen Evropa ale podepsalo jí i hodně jiných států, mezinárodně
nejspupěšnější
    \item Regionální úmluvy - platí pouze na určitém místě - lokální dohody (teritoriálně) - částečně
kompatibilní - Dohoda o spolupráci při boji proti kriminalitě souvisejícé s počítačovými daty
(Commonwealth, 2001) - Dohoda o boji proti IT trestných činech (Arabská liga, 2010) -
Dohoda o spolupráci v oblasti mezinárodní informační bezpečnosti (Shanghajská organizace
spolupráce, 2010) - Úmluva o kybernetické bezpečnosti a ochraně osobních údajů (Africká
unie, 2014)
    \item Dvoustranné úmluvy - mezi dvěma státy, bilaterární (tipicky mezi sousedními) - v
kyberkriminalitě se to moc nepoužívá
    \item Vzorové právo - nástroj který není nijak vymahatelná, nikdo se k ničemu nezavazuje - stát v
určitých regionech často nemá prostředky na vytvoření vhodné právní regulace určité oblasti
- nadnárodní celky vytváří vrozové právo nebo nějaké řešení určité oblasti (problému),
vpodstě říkají tady problém a my si myslíme že by se to měl řešit takhle, a vy to mužete a
nemusí příjmout - státy mohou a nemusí tyto upravy příjmout - Vzorové právo pro
počítačovou kriminalitu a byberkriminalitu (Jihoafrická společnost pro spolupráci) - OSN resp.
UNODC se snažilo o přípravu ale připravit i vzorové právo je složité a proto jich je spoustu
rozpracovaných ale nedokončených
    \item Právo EU - velmi specifické postavení - reguluje se regionálně - je to závazné pro všechny v
daném regionu pomocí nařízení a směrnic - Směrnice o útocích na informační systémy
(2013), z hlediska hmotně právní úpravy podobná Úmluvě o kyberkriminalitě
\end{itemize}

\subsection{Snahy do budoucna}
OSN se snaží ale jak je velká a má obrovský dopad tak je problém něco prosadit - jak má hodně
členu tak se najdou ty co do toho hází vidle (čína, rusko - jim se tento stav libí a nechtějí ho měnit,
vpodstatě říkají: nepříjmeme umluvu o kyberkriminalitě protože je považuejme za lokální
evropskou úmluvu, chceme vymyslet nové právo ale budem vám do toho házet vidle do té doby
než se na to zapomene) - nepodařilo se příjmout ani studii o z mapování páchání
kyberkriminality po světě (zůstalo to ve formě návrhu) - to se ostatním státum nelíbí a snaží se
tedy co nejvíc rozšířít svojí jurisdikci (predevším EU a USA) - např GDPR (nařízeni o ochraně
osobních údaju) má velmi specifickou jurisdikci, vztahuje se na všechny subjekty ať jsou usídleny
kdekoliv které poskytují služby v EU (vztahuje se to i na čínu a rusko), druhá otázka je pak
vymahatelnost techto pravidel - EU se to líbí a začínají to uplatňovat i do ostatních uprav - USA
příjmuli CLOUD act, který říká že je jedno kde máš dat o amerických občanech, ale pokud je to v
zahraničí tak má povinnost je vydat americkým úřadum (při požádáni i quess??)
Do určité miri dochází k fragmentaci internetu (rusko a čína to velmi podporují) - čína si striktně
dozoruje co jde dovnitř a ven z jejich internetu (velký čínský firewall), např data o čínskách
občanech se nesmí uchovávat mino čínu - rusko vyhrožuje totálním odpojením jejich části sítě

\subsection{Procesní právo a spolupráce}
Pokud někdo nechce spolupracovat existují nástroje které se dají ohnout protřeby
kyberkriminality - OČTŘ musí požádat jiný stát o spolupráci - existují ruzné mechanizmi jak se
tohoto dosahuje - jeden zpusob je stát obejít a rovnou se to řeší s poskytovately konkretních
služeb - obecně mívají i více lidí kteří mají ke spolupráci dostatek znalostí + vetšinou jsou to stejně
oni kdo drží ty data 
\begin{itemize}
    \item formální spolupráce států - MLA (Mutual Legal Assistance) mezinárodní justiční pomoc -
nejstandardnější mechanizmus upravený v mezinárodních smlouvách a ve většině právních
řádů - umožňuje OČTR sepsat žádost o mezinárodní justiční spolupráci která se předá
pomocí komunikujících zastupitelských úřadů a čeká se než dožádaný stát poskytně
spolupráci - problém je že je to velmi pomalé (enormí birokracie) a stát na to nemusí vyhovět
- jeden z požadavků je aby byla dostupné procesní nástroje v obou státech - v EU je pak dálší
způsob -MR (Mutual Recognition) vzájemné uznávání rozhodnutí - fungující specificky v
EU - na základě Evropského vyšetřovacího příkazu - uplatňuje se u některých procesních
nástrojů, které upravují evropské předpisy - dožádaný stát musí příjmou (s extrémě uzkými
vyjímkami) rozhodnutí jiného státu a s určitou prioritou ho musí vykonat - jsou zde lhůty(30
dní na rozhodnutí + 90 na vykonání příkazu) - odpadá část birokracia - a povynnost
součinnosti - pak ještě exitují Dvoustranné nástroje - dva státy se dohodnou na lepší
spulupráci - většinou státy které mají k sobě blízko - většinou né v kyberkriminalitě - má me
třeba se sousedníma státama že když někoho stíhají v autě a přejedou hranice tak můžou
pokračovat
    \item Neformální spolupráce států - orgány nebo lidi se znají a tak si poskytnou spolupráci nebo se
aspoň asměrují na správnou cestu/směr - konzultace, expertázy, vybavení - může to
koordinovat Europol nebo Interpol nebo může být zprostředkována přes bezpečnostní týmy
nebo jiné státní orgány nebo sdružení např sdružení energetických společností - neformálni
cesty negenerují tak kvalitní důkazy - na druhou stranu je to o dost rychlejí - pokud třeba jen
zjišťujeme kde se důkazy mohou nacházek kdo je může mít, jaký poskytovatel tak se můžeme
jednoduše takhle neformálně doptat jiných států/orgánu/společností (na tohle je to super)
    \item Prímá spolupráce s ISP - Existující právní regulace - Preservation and production směrnice
(EU) - Cloud act (US) - nebo dobrovolná spolupráce
\end{itemize}
schéma ukazující spolupráci dvou států - zdlouhavé, vůži volatilitě el. důkazů - někdy třeba státy
nevyhoví vůbec nebo třeba za 2 roky, kdy už je to jedno
nástroje které definuje Úmluva o kyberkriminalitě viz. odkaz
Evropský vyšetřovací příkaz - postaven na MR - vzájemné uznávání rozhodnutí - měl by přinést
vetší efektivitu při mezinárodním zajišťování důkazů - nastavené lhůty (po přijetí příkazu 30 dnů
na rozhodnutí jestli je nebo není proveditelný a do 90 dnu pak vykonat v případě že jde vykonat) -
stejná priorita cizích rozhodnutí jako lokálních rozhodnutí - omezení kdy tomu může státat
nevyhovět (protiustavní apod. hodně limitované)
Přikáz k zachování a vydání data - zatím nepřijatá směrnice EU - stát by mohl vyžadovat přimo
po ISP v cizí zemi vydání dat - platilo by pro ISP poskytující služby v EU - rozšiřuje to jurisdikci na
všechny které poskytují služby
CLOUD act - USA - americké orgány mohou přistupovat k datům spravovaných o amerických
subjekterch i mimo území USA - google bude mít data o američanech uložená na serveru v EU ->
má povinnost tady data vydat americkým orgánům - pro nás je zajímavá úprava executive
agreements (výkonné smlouvy), které se uzavírají s určitýma regionama ve světě - jedná se o
uzavření mezi US a EU, zatím neexistuje ale je snaha aby tomu tak bylo - tak by vznikla možnost
vyžadování dat i opačně (po amerických poskytovatelech služeb, které by drželi udaje o češích)

\subsection{Aktuální iniciativa}
\begin{itemize}
    \item UNODC - snažili se o vzorové právo, studii, která nakonec nebyla přijata - vytvořili portál kde
se sdílí soudní rozhodnutí a právo ohledně kyberkriminality pro informační důvody - dále
dělají vzdělávací moduly na organizovaný zločin všetně kyberkriminality (společné vzdělávání
vyšetřovatelů, nebo rovojových zemí aby třeba somálsko mělo někoho kdo dokáže
poskytnou spolupráci třeba) - takové měkčí iniciatiy
    \item Rada Evropy - tvrdší charakter než snahy UNODC - skupina odborníků která pracuje na
druhém dodatkovém protokolu k Úmluvě o kyberkriminalitě - který by měl řešit
problém s omezenou možností spolupráce (nové procesní nástroje pro exektivní předávání
dat, videokonference, mechanismus předávání dat apod)
    \item EU - vytváří certifikaní mechanismus nástrojů pro vyšetřování kyberkriminality a nástroju
předávání el. důkazů a pracují daných mechanismech na elektonickém předávání důkazního
materiálu - neexistuje jednoznačný postup pro předání el. důkazu (někdy se převází na
flashce autem, někdy nějak elektronicky - stím ale můžou mít soudy problém že to není
bezpečné)
\end{itemize}
Dále pomáhají taxonomie pro kategorizaci kyberkriminality - Europská taxonomie
(ENISA/EUROPOL) - společná klasifikace bezpečnostních incidentů a jejich navázání na europskou
právní úpravu. Lepší statiky než kdyby se klasifikovalo na základě TP hmotného - OSN taxonomie -
UNODC
